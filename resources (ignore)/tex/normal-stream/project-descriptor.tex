% Author: Grayson Orr
% Course: ID721001: Mobile Application Development

\documentclass{article}
\author{}

\usepackage{graphicx}
\usepackage{wrapfig}
\usepackage{enumerate}
\usepackage{hyperref}
\usepackage[margin = 2.25cm]{geometry}
\usepackage[table]{xcolor}
\usepackage{fancyhdr}
\hypersetup{
  colorlinks = true,
  urlcolor = blue
}
\setlength\parindent{0pt}
\pagestyle{fancy}
\fancyhf{}
\rhead{College of Engineering, Construction and Living Sciences\\Bachelor of Information Technology}
\lfoot{Project\\Version 1, Semester Two, 2023}
\rfoot{\thepage}
 
\begin{document}

\begin{figure}
	\centering
	\includegraphics[width=50mm]{"../../../resources (ignore)/img/logo.png"}
\end{figure}
\title{College of Engineering, Construction and Living Sciences\\Bachelor of Information Technology\\ID721001: Mobile Application Development\\Level 7, Credits 15\\\textbf{Project}}
\date{}
\maketitle

\section*{Assessment Overview}
In this \textbf{individual} assessment, you will develop two mobile applications using \textbf{React Native} and \textbf{Expo}, and publish the second mobile application to either \textbf{Google Play Store} or \textbf{Apple App Store}. In addition, marks will be allocated for code elegance, documentation and \textbf{Git/GitHub} usage. For Part 3, you will research, prepare and present a \textbf{React Native} UI library. The information presented must be in a \textbf{README.md} file. Also, you need to provide code examples to accompany the \textbf{README.md} file. The main purpose of this assessment is to demonstrate your ability to identify and effectively articulate your findings.\\

\section*{Learning Outcomes}
At the successful completion of this course, learners will be able to:
\begin{enumerate}
	\item Implement and publish complete, non-trivial, industry-standard mobile applications following sound architectural and code-quality standards.
	\item Identify relevant use cases for a mobile computing scenario and incorporate them into an effective user experience design.
	\item Follow industry standard software engineering practice in the design of mobile applications.
\end{enumerate}

\section*{Assessments}
\renewcommand{\arraystretch}{1.5}
\begin{tabular}{|c|c|c|c|}
	\hline
	\textbf{Assessment} & \textbf{Weight} & \textbf{Due Date}    & \textbf{Learning Outcomes} \\ \hline
	Practical          & 20\%            & 22-09-2023 (Friday at 4.59 PM)  & 2, 3                    \\ \hline
	Project           & 80\%            & Various  & 1, 2, 3                    \\ \hline
\end{tabular}

\section*{Conditions of Assessment}
You will complete majority of this assessment during your learner-managed time. However, there will be time during class to discuss the requirements and your progress on this assessment. Part 1 will need to be completed by \textbf{Friday, 01 September 2023} at \textbf{4.59 PM} and Part 2 and 3 will need to be completed by \textbf{Friday, 10 November 2023} at \textbf{4.59 PM}.

\section*{Pass Criteria}
This assessment is criterion-referenced (CRA) with a cumulative pass mark of \textbf{50\%} over all assessments in \textbf{ID721001: Mobile Application Development}.

\section*{Authenticity}
All parts of your submitted assessment \textbf{must} be completely your work. Do your best to complete this assessment without using an \textbf{AI generative tool}. You need to demonstrate to the course lecturer that you can meet the learning outcome(s) for this assessment. \\
 
 However, if you get stuck, you can use an \textbf{AI generative tool} to help you get unstuck, permitting you to acknowledge that you have used it. In the assessment's repository \textbf{README.md} file, please include what prompt(s) you provided to the \textbf{AI generative tool} and how you used the response(s) to help you with your work. It also applies to code snippets retrieved from \textbf{StackOverflow} and \textbf{GitHub}. \\
 
 Failure to do this may result in a mark of \textbf{zero} for this assessment. 

\section*{Policy on Submissions, Extensions, Resubmissions and Resits}
The school's process concerning submissions, extensions, resubmissions and resits complies with \textbf{Otago Polytechnic} policies. Learners can view policies on the \textbf{Otago Polytechnic} website located at \href{https://www.op.ac.nz/about-us/governance-and-management/policies}{https://www.op.ac.nz/about-us/governance-and-management/policies}.

\section*{Submission}
You \textbf{must} submit all project files via \textbf{GitHub Classroom}. Here are the URLs to the repositories you will use for your submission:

\begin{itemize}
	\item Part 1 - \href{https://classroom.github.com/a/pWdicmvU}{https://classroom.github.com/a/pWdicmvU}
	\item Part 2 - \href{https://classroom.github.com/a/d\_u8vXR5}{https://classroom.github.com/a/d\_u8vXR5}
\end{itemize}

Create a \textbf{.gitignore} and add the ignored files in this resource - \href{https://raw.githubusercontent.com/github/gitignore/main/Node.gitignore}{https://raw.githubusercontent.com/github/gitignore/main/Node.gitignore}. The latest project files in the \textbf{master} or \textbf{main} branch will be used to mark against the \textbf{Functionality} criterion. Please test before you submit. Partial marks \textbf{will not} be given for incomplete functionality. Late submissions will incur a \textbf{10\% penalty per day}, rolling over at \textbf{5:00 PM}.

\section*{Extensions}
Familiarise yourself with the assessment due date. If you need an extension, contact the course lecturer before the due date. If you require more than a week's extension, a medical certificate or support letter from your manager may be needed.

\section*{Resubmissions}
Learners may be requested to resubmit an assessment following a rework of part/s of the original assessment. Resubmissions are to be completed within a negotiable short time frame and usually \textbf{must} be completed within the timing of the course to which the assessment relates. Resubmissions will be available to learners who have made a genuine attempt at the first assessment opportunity and achieved a \textbf{D grade (40-49\%)}. The maximum grade awarded for resubmission will be \textbf{C-}.

\section*{Resits}
Resits and reassessments \textbf{are not} applicable in \textbf{ID721001: Mobile Application Development}.

\section*{Instructions}
You will need to submit a mobile application and documentation that meet the following requirements:

\section{Part 1: Cookbook Application (20\%)}

\subsection*{Functionality - Learning Outcomes 1, 2, 3 (50\%)}
\begin{itemize}
  \item The mobile application needs to run without code or file structure modification in \textbf{Visual Studio Code}.
	\item Usable on a variety of mobile devices, i.e., devices with different screen sizes.
	\item Free of bugs that significantly effect the usability.
	\item Food data needs to be fetched using \textbf{Axios} from a \textbf{GitHub Gist}. You have been provided an example file called \textbf{food-data.json}.
	\item Display \textbf{bottom tab navigation} with the following screens:
	\begin{itemize}
    \item Daily specials
    \begin{itemize}
      \item This screen will display six random recipes from the \textbf{food-data.json} file. 
      \item Display the random recipes in a \textbf{FlatList}.
      \item Each recipe item in the \textbf{FlatList} needs to display the recipe's name and image. Truncate the recipe's name if it is too long.
      \item When a recipe item is pressed, display the recipe's name, image, cuisine, ingredients and instructions in a \textbf{ScrollView}. 
    \end{itemize}
    \item Recipes
    \begin{itemize}
      \item This screen will display all cuisines from the \textbf{food-data.json} file.
      \item When a cuisine item is pressed, display all recipes for that cuisine in a \textbf{FlatList}.
      \item Each recipe item in the \textbf{FlatList} needs to display the recipe's name, description and image. Truncate the recipe's name and description if it is too long.
      \item When a recipe item is pressed, display the recipe's name, image, cuisine, ingredients and instructions in a \textbf{ScrollView}. 
    \end{itemize}
    \item A heart icon needs to be displayed in the top right corner of the screen. When the heart icon is pressed, the recipe is added to the \textbf{Favourites} screen. Persist the favourite recipes using \textbf{AsyncStorage}.
    \item A plus icon needs to be display next to the heart icon. When the plus icon is pressed, the recipe's ingredients are added to the \textbf{Shopping list} screen. Persist the shopping list using \textbf{AsyncStorage}.
    \item Favourites
    \begin{itemize}
      \item This screen will display all recipes that have been added to the \textbf{Favourites} screen.
      \item Display the favourite recipes stored in \textbf{AsyncStorage} in a \textbf{FlatList}. 
      \item Ability to delete a favourite recipe from the \textbf{FlatList}.
    \end{itemize}
    \item Shopping list
    \begin{itemize}
      \item This screen will display all ingredients from the recipes that have been added to the \textbf{Shopping list} screen.
      \item Display the shopping list stored in \textbf{AsyncStorage} in a \textbf{FlatList}.
      \item Ensure there are no duplicate ingredients in the \textbf{FlatList}.
      \item Ability to delete an ingredient from the \textbf{FlatList}.
    \end{itemize}
    \item Appropriate image used for the splash screen and app icon.
    \item Visually attractive UI with a coherent graphical theme and style.
  \end{itemize}
\end{itemize}

\subsection*{Code Elegance - Learning Outcomes 1, 3 (40\%)}
\begin{itemize}
	\item A \textbf{Node.js} \textbf{.gitignore} file is used.
	\item If applicable, a \textbf{.env} and \textbf{.env.example} file is used.
  \item Appropriate naming of files, variables, functions and components.
	\item Idiomatic use of control flow, data structures and in-built functions.
  \item Efficient algorithmic approach.
  \item Sufficient modularity.
  \item Each \textbf{component} file \textbf{must} have a \textbf{JSDoc} header comment located immediately before the \textbf{import} statements.
\item In-line comments where required. It should be for code that needs further explanation.
  \item Code is formatted.
\item No dead or unused code. 
\end{itemize}

\subsection*{Documentation and Git/GitHub Usage - Learning Outcomes 2, 3 (10\%)}
\begin{itemize}
	\item \textbf{GitHub} project board to help you organise and prioritise your work. 
    \item Provide the following in your repository \textbf{README.md} file:
    \begin{itemize} 
      \item Wireframes of the mobile application's screens. The wireframes can be either hand-drawn or created using a digital tool.
	  \item How do you setup the environment, i.e., after the repository is cloned?
      \item How do you format your code?
      \item If applicable, known bugs.
    \end{itemize}
    \item Use of \textbf{Markdown}, i.e., headings, bold text, code blocks, etc.
    \item Correct spelling and grammar.
    \item Your \textbf{Git commit messages} should:
    \begin{itemize}
      \item Reflect the context of each functional requirement change.
      \item Be formatted using an appropriate naming convention style.
    \end{itemize}
\end{itemize}

\section{Part 2: Traveling Application (40\%)}

\textbf{Scenario:} The mobile application will help you sound like a local and adapt to a new culture. You will begin by selecting a country you wish to travel to. For example, if you wish to travel to Italy, you would be provided with all the necessary tools such as text translation and text to speech support, a selection of well-known Italian phrases and a map containing locations of Italy's top-rated tourist attractions. A user of your mobile application \textbf{must} be able to select from at least \textbf{six} countries with at least \textbf{one} country per \href{https://www.worldometers.info/geography/7-continents/}{continent} \textbf{excluding} Antarctica.

\subsection*{Functionality - Learning Outcomes 1, 2, 3 (50\%)}
\begin{itemize}
	\item The mobile application needs to run without code or file structure modification in \textbf{Visual Studio Code}.
	\item Usable on a variety of mobile devices, i.e., devices with different screen sizes.
	\item Free of bugs that significantly effect the usability.
	\item The mobile application is published to \textbf{Google Play Store} or \textbf{Apple App Store}.
	\begin{itemize}
		\item To published to \textbf{Google Play Store} or \textbf{Apple App Store}, you will need an account. The account's credentials will be privately given to you on \textbf{Microsoft Teams}. \textbf{Do not} disable any applications published on this account.
	\end{itemize}
	\item Ability to download the mobile application from \textbf{Google Play Store} or \textbf{Apple App Store} on to a variety of mobile devices.
	\item Country data needs to be fetched using \textbf{Axios} from a \textbf{GitHub Gist}. You have been provided an example file called \textbf{country-data.json}. \textbf{Note:} This file is incomplete. You will need to add more information to the file. 
	\item Display a list of countries in a \textbf{FlatList}. Each item needs to have the country's name and flag. When you select a country, it will navigate to the country's screen containing the following:
	\begin{itemize}
		\item Text translation support. If a country is multilingual (use of more than one language), choose one language. For example, Canada's main languages are English and French. You would choose either English or French.
		\begin{itemize}
			\item Use \textbf{Axios} and the \textbf{Yandex Translate API} to translate text from one language to another. To use the \textbf{Yandex Translate API}, you will need an \textbf{API key}. A key will be privately given to you on \textbf{Microsoft Teams}.
			\item Display some feedback while the text is being translated.
			\item Handle incorrectly formatted input fields. For example, an \textbf{TextInput} is blank or empty.
		\end{itemize}
		\item Text to speech support.
			\begin{itemize}
				\item Handle incorrectly formatted input fields. For example, an \textbf{TextInput} is blank or empty, or the country is not supported.
			\end{itemize}
		\item Selection of two well-known phrases. For example, "No worries, mate, she'll be right" is well-known phrase in Australia. 
		\item \textbf{Google Map} displaying top-rated tourist attractions.
		\begin{itemize}
			\item Display two tourist attractions per continent.
			\item Each data object will represent a marker.
			\item The marker's information window needs to display the attraction's name and city/town.
		\end{itemize}
  		\item Display an image gallery of top-rated tourist attractions. Maximum of four images per attraction.
  		\item A \textbf{WebView} containing the country's Wikipedia page.
  		\item An interactive quiz for each country.
		  \begin{itemize}
			  \item Quiz topics may include animals, culture, food, drink, geography and sport.
			  \item Each quiz needs to have five questions.
			  \item Questions are multi-choice and need to have four answers.
			  \item Each question needs to have an image.
			  \item Display appropriate feedback in a \textbf{TextInput} when a question is answered correctly or incorrectly. If an answer is incorrect, display
			   the correct answer.
			  \item At the end the quiz, display the score in a \textbf{TextInput} and store the score in \textbf{Firestore} on \textbf{Firebase}.
		  \end{itemize}  
	\end{itemize}
	\item Appropriate image used for the splash screen and app icon.
	\item Visually attractive UI with a coherent graphical theme and style.
\end{itemize}

\subsection*{Code Elegance - Learning Outcomes 1, 3 (40\%)}
\begin{itemize}
	\item A \textbf{Node.js} \textbf{.gitignore} file is used.
	\item If applicable, a \textbf{.env} and \textbf{.env.example} file is used.
  \item Appropriate naming of files, variables, functions and components.
	\item Idiomatic use of control flow, data structures and in-built functions.
  \item Efficient algorithmic approach.
  \item Sufficient modularity.
  \item Each \textbf{component} file \textbf{must} have a \textbf{JSDoc} header comment located immediately before the \textbf{import} statements.
	\item In-line comments where required. It should be for code that needs further explanation.
  \item Code is formatted.
\item No dead or unused code. 
\end{itemize}

\subsection*{Documentation and Git/GitHub Usage - Learning Outcomes 2, 3 (10\%)}
\begin{itemize}
	\item \textbf{GitHub} project board to help you organise and prioritise your work. 
    \item Provide the following in your repository \textbf{README.md} file:
    \begin{itemize} 
	  \item Link to the mobile game on \textbf{Google Play Store} or \textbf{Apple App Store}.
      \item Wireframes of the mobile application's screens. The wireframes can be either hand-drawn or created using a digital tool.
	  \item How do you setup the environment, i.e., after the repository is cloned?
      \item If applicable, known bugs.
    \end{itemize}
    \item Use of \textbf{Markdown}, i.e., headings, bold text, code blocks, etc.
    \item Correct spelling and grammar.
    \item Your \textbf{Git commit messages} should:
    \begin{itemize}
      \item Reflect the context of each functional requirement change.
      \item Be formatted using an appropriate naming convention style.
    \end{itemize}
\end{itemize}

\section{Part 3: Presentation (20\%)}

List of UI libraries:

\begin{itemize}
	\item NativeBase
	\item Teaset
	\item Material Kit React Native
	\item React Native Elements
	\item React Native Paper
	\item Lottie for React Native
	\item Reatct Native UI Kitten
	\item React Native Material Kit
	\item React Native Material UI
	\item RNUILib
\end{itemize}

\subsection*{Documentation - Learning Outcomes 2, 3 (50\%)}
\begin{itemize}
	\item Documentation must contain the following sections:
	      \begin{itemize}
		      \item Overview - a brief description of what the UI library is.
		      \item How to install - a description of how to install the UI library.
		      \item Code examples - provide five code examples of how to use the UI library.
		      \item References - the information in your documentation is referenced using \textbf{APA 7th edition}.
		            \begin{itemize}
			            \item \textbf{Resource:} \href{https://studentservices.op.ac.nz/learning-support/citingandreferencing}{https://studentservices.op.ac.nz/learning-support/citingandreferencing}
		            \end{itemize}
	      \end{itemize}
	\item Use of \textbf{Markdown}, i.e., bold text, code blocks, etc.
	\item Correct spelling and grammar.
\end{itemize}

\subsection*{Presentation 2, 3 (50\%)}
\begin{itemize}
	\item Present your documentation, i.e., \textbf{README.md} via a video recording. In addition, you \textbf{must}:
	      \begin{itemize}
		      \item Upload your presentation to your \textbf{OneDrive}.
		      \item Provide a link to your presentation in your documentation.
	      \end{itemize}
\end{itemize}

\subsection*{Additional Information}
\begin{itemize}
	\item \textbf{Do not} rewrite your \textbf{Git} history. It is important that the course lecturer can see how you worked on your assessment over time.
	\item You need to provide the wireframes to the course lecturer before you begin development.
	\item The presentation must not exceed \textbf{30 minutes} in length.
	\item Upload your presentation to \textbf{OneDrive}. Email a link to your presentation to \href{grayson.orr@op.ac.nz}{grayson.orr@op.ac.nz}.
\end{itemize}

\end{document}