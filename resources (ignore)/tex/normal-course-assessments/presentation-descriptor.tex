% Author: Grayson Orr
% Course: ID721001: Mobile Application Development

\documentclass{article}
\author{}

\usepackage{graphicx}
\usepackage{wrapfig}
\usepackage{enumerate}
\usepackage{hyperref}
\usepackage[margin = 2.25cm]{geometry}
\usepackage[table]{xcolor}
\usepackage{fancyhdr}
\hypersetup{
  colorlinks = true,
  urlcolor = blue
}
\setlength\parindent{0pt}
\pagestyle{fancy}
\fancyhf{}
\rhead{College of Engineering, Construction and Living Sciences\\Bachelor of Information Technology}
\lfoot{Presentation\\Version 3, Semester Two, 2023}
\rfoot{\thepage}
 
\begin{document}

\begin{figure}
	\centering
	\includegraphics[width=50mm]{"../../../resources (ignore)/img/logo.png"}
\end{figure} 

\title{College of Engineering, Construction and Living Sciences\\Bachelor of Information Technology\\ID721001: Mobile Application Development\\Level 7, Credits 15\\\textbf{Presentation}}
\date{}
\maketitle

\section*{Assessment Overview}
In this \textbf{individual} assessment, you will research, prepare and present a \textbf{React Native} UI library. The information presented must be in a \textbf{README.md} file. Also, you need to provide code examples to accompany the \textbf{README.md} file. The main purpose of this assessment is to demonstrate your ability to identify and effectively articulate your findings.

\section*{Learning Outcomes}
At the successful completion of this course, learners will be able to:
\begin{enumerate}
	\item Implement and publish complete, non-trivial, industry-standard mobile applications following sound architectural and code-quality standards.
	\item Identify relevant use cases for a mobile computing scenario and incorporate them into an effective user experience design.
	\item Follow industry standard software engineering practice in the design of mobile applications.
\end{enumerate}

\section*{Assessments}
\renewcommand{\arraystretch}{1.5}
\begin{tabular}{|c|c|c|c|}
	\hline
	\textbf{Assessment} & \textbf{Weight} & \textbf{Due Date}    & \textbf{Learning Outcomes} \\ \hline
	Project 1: Cookbook Application            & 20\%            & 01-09-2023 (Friday at 4.59 PM)  & 1, 2, 3                    \\ \hline
	Project 2: Travelling Application            & 40\%            & 10-11-2023 (Friday at 4.59 PM)  & 1, 2, 3                    \\ \hline
	Practical: Skills-Based           & 20\%            & 22-09-2023 (Friday at 4.59 PM)  & 2, 3                    \\ \hline
	Presentation       & 20\%            & 10-11-2023 (Friday at 4.59 PM) & 2, 3                       \\ \hline
\end{tabular}

\section*{Conditions of Assessment}
You will complete this assessment during your learner-managed time. However, there will be time during class to discuss the requirements and your progress on this assessment. This assessment will need to be completed by \textbf{Friday, 10 November 2023} at \textbf{4.59 PM}.

\section*{Pass Criteria}
This assessment is criterion-referenced (CRA) with a cumulative pass mark of \textbf{50\%} over all assessments in \textbf{ID721001: Mobile Application Development}.

\section*{Authenticity}
All parts of your submitted assessment \textbf{must} be completely your work. Do your best to complete this assessment without using an \textbf{AI generative tool}. You need to demonstrate to the course lecturer that you can meet the learning outcome(s) for this assessment. \\
 
 However, if you get stuck, you can use an \textbf{AI generative tool} to help you get unstuck, permitting you to acknowledge that you have used it. In the assessment's repository \textbf{README.md} file, please include what prompt(s) you provided to the \textbf{AI generative tool} and how you used the response(s) to help you with your work. It also applies to code snippets retrieved from \textbf{StackOverflow} and \textbf{GitHub}. \\
 
 Failure to do this may result in a mark of \textbf{zero} for this assessment.

\section*{Policy on Submissions, Extensions, Resubmissions and Resits}
The school's process concerning submissions, extensions, resubmissions and resits complies with \textbf{Otago Polytechnic} policies. Learners can view policies on the \textbf{Otago Polytechnic} website located at \href{https://www.op.ac.nz/about-us/governance-and-management/policies}{https://www.op.ac.nz/about-us/governance-and-management/policies}.

\section*{Extensions}
Familiarise yourself with the assessment due date. If you need an extension, contact the course lecturer before the due date. If you require more than a week's extension, a medical certificate or support letter from your manager may be needed.

\section*{Resubmissions}
Learners may be requested to resubmit an assessment following a rework of part/s of the original assessment. Resubmissions are to be completed within a negotiable short time frame and usually \textbf{must} be completed within the timing of the course to which the assessment relates. Resubmissions will be available to learners who have made a genuine attempt at the first assessment opportunity and achieved a \textbf{D grade (40-49\%)}. The maximum grade awarded for resubmission will be \textbf{C-}.

\section*{Resits}
Resits and reassessments \textbf{are not} applicable in \textbf{ID721001: Mobile Application Development}.

\section*{Instructions}

List of UI libraries:

\begin{itemize}
	\item NativeBase
	\item Teaset
	\item Material Kit React Native
	\item React Native Elements
	\item React Native Paper
	\item Lottie for React Native
	\item Reatct Native UI Kitten
	\item React Native Material Kit
	\item React Native Material UI
	\item RNUILib
\end{itemize}

\subsection*{Documentation - Learning Outcomes 2, 3 (50\%)}
\begin{itemize}
	\item Documentation must contain the following sections:
	      \begin{itemize}
		      \item Overview - a brief description of what the UI library is.
		      \item How to install - a description of how to install the UI library.
		      \item Code examples - provide five code examples of how to use the UI library.
		      \item References - the information in your documentation is referenced using \textbf{APA 7th edition}.
		            \begin{itemize}
			            \item \textbf{Resource:} \href{https://studentservices.op.ac.nz/learning-support/citingandreferencing}{https://studentservices.op.ac.nz/learning-support/citingandreferencing}
		            \end{itemize}
	      \end{itemize}
	\item Use of \textbf{Markdown}, i.e., bold text, code blocks, etc.
	\item Correct spelling and grammar.
\end{itemize}

\subsection*{Presentation 2, 3 (50\%)}
\begin{itemize}
	\item Present your documentation, i.e., \textbf{README.md} via a video recording. In addition, you \textbf{must}:
	      \begin{itemize}
		      \item Upload your presentation to your \textbf{OneDrive}.
		      \item Provide a link to your presentation in your documentation.
	      \end{itemize}
\end{itemize}

\subsection*{Additional Information}
\begin{itemize}
	\item The presentation must not exceed \textbf{30 minutes} in length.
\end{itemize}
\end{document}
