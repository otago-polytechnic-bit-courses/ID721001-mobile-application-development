% Author: Grayson Orr
% Course: IN721: Mobile Application Development

\documentclass{article}
\author{}

\usepackage{graphicx}
\usepackage{wrapfig}
\usepackage{enumerate}
\usepackage{hyperref}
\usepackage[margin = 2.25cm]{geometry}
\usepackage[table]{xcolor}
\usepackage{fancyhdr}
\hypersetup{
  colorlinks = true,
  urlcolor = blue
}
\setlength\parindent{0pt}
\pagestyle{fancy}
\fancyhf{}
\rhead{College of Engineering, Construction and Living Sciences\\Bachelor of Information Technology}
\lfoot{Project\\Version 1, Summer School, 2021-2022}
\rfoot{\thepage}
 
\begin{document}

\begin{figure}
	\centering
	\includegraphics[width=50mm]{../../resources/img/logo.png}
\end{figure}

\title{College of Engineering, Construction and Living Sciences\\Bachelor of Information Technology\\IN721: Mobile Application Development\\Level 7, Credits 15\\\textbf{Project}}
\date{}
\maketitle

\section*{Assessment Overview}
In this assessment, you will develop \& publish a travelling application using \textbf{Kotlin} in \textbf{Android Studio} \& \textbf{Google Play Store}. \textbf{Android} topics such as \textbf{ViewModel}, \textbf{LiveData}, \textbf{Room Database} \& \textbf{Google Map} were formally covered in the teaching sessions. The main purpose of this assessment is not just to build a mobile application, rather to demonstrate your ability to effectively implement intermediate/advanced \textbf{Android} features \& other application development topics. In addition, marks will be allocated for code elegance, documentation \& \textbf{Git/GitHub} usage. \\

The travelling application will help you sound like a local \& adapt to a new culture. You will begin by selecting a country you wish to travel to. For example, if you wish to travel to Italy, you would be provided with all the necessary tools such as text translation \& text to speech support, a selection of well-known Italian phrases, an interactive quiz to test your knowledge of Italian culture \& a map containing locations of Italy's top-rated tourist attractions. A user of your travelling application \textbf{must} be able to select from at least \textbf{six} countries with at least \textbf{one} country per \href{https://www.worldometers.info/geography/7-continents/}{continent} \textbf{excluding} Antarctica. 

\section*{Learning Outcomes}
At the successful completion of this course, learners will be able to:
\begin{enumerate}
	\item Implement \& publish complete, non-trivial, industry-standard mobile applications following sound architectural \& code-quality standards.
	\item Identify relevant use cases for a mobile computing scenario \& incorporate them into an effective user experience design.
	\item Follow industry standard software engineering practice in the design of mobile applications.
\end{enumerate}

\section*{Assessment Table}
\renewcommand{\arraystretch}{1.5}
\begin{tabular}{|l|l|l|l|l|}
	\hline
	\vtop{\hbox{\strut \textbf{Assessment}}\hbox{\strut \textbf{Activity}}} & \textbf{Weighting} & \vtop{\hbox{\strut \textbf{Learning}}\hbox{\strut \textbf{Outcomes}}} & \vtop{\hbox{\strut \textbf{Assessment}}\hbox{\strut \textbf{Grading Scheme}}} & \vtop{\hbox{\strut \textbf{Completion}}\hbox{\strut \textbf{Requirements}}} \\
	
	\hline                                              
	\small Project                                                          & \small 80\%        & \small 1, 2, 3                                                        & \small CRA                                                                    & \small Cumulative                                                           \\ \hline
	\small Presentation                                                          & \small 20\%        & \small 2, 3                                                        & \small CRA                                                                    & \small Cumulative                                                           \\ \hline
\end{tabular}

\section*{Conditions of Assessment}
You will complete this assessment during your learner managed time, however, there will be availability during the weekly meetings to discuss the requirements \& your progress of this assessment. This assessment is due on \textbf{11/02/2022} at \textbf{5:00 PM}.

\section*{Pass Criteria}
This assessment is criterion-referenced (CRA) with a cumulative pass mark of \textbf{50\%} over all assessments in \textbf{IN721: Mobile Application Development}.

\section*{Authenticity}
All parts of your submitted assessment \textbf{must} be completely your work \& any references \textbf{must} be cited appropriately including, externally-sourced graphic elements using \textbf{APA 7th edition}. Provide your references in a \textbf{README.md} file. All media \textbf{must} be royalty free (or legally purchased) for educational use. Failure to do this will result in a mark of \textbf{zero} for this assessment.

\section*{Policy on Submissions, Extensions, Resubmissions \& Resits}
The school's process concerning submissions, extensions, resubmissions \& resits complies with \textbf{Otago Polytechnic} policies. Learners can view policies on the \textbf{Otago Polytechnic} website located at \href{https://www.op.ac.nz/about-us/governance-and-management/policies}{https://www.op.ac.nz/about-us/governance-and-management/policies}.

\section*{Submissions}
You \textbf{must} submit all program files via \textbf{GitHub Classroom}. Here is the URL to the repository you will use for your submission – \href{https://classroom.github.com/a/YvRoxlAh}{https://classroom.github.com/a/YvRoxlAh}. The latest program files in the \textbf{main} branch will be used to mark against the \textbf{Functionality} criterion. Please test your \textbf{main} branch application before you submit. Partial marks \textbf{are not} given for functionality in other branches. Late submissions will incur a \textbf{10\% penalty per day}, rolling over at \textbf{5:00 PM}.

\section*{Extensions}
Familiarise yourself with the assessment due date. If you need an extension, contact the course lecturer before the due date. If you require more than a week's extension, a medical certificate or support letter from your manager may be needed.

\section*{Resubmissions}
Learners may be requested to resubmit an assessment following a rework of part/s of the original assessment. Resubmissions are to be completed within a negotiable short time frame \& usually \textbf{must} be completed within the timing of the course to which the assessment relates. Resubmissions will be available to learners who have made a genuine attempt at the first assessment opportunity \& achieved a \textbf{D grade (40-49\%)}. The maximum grade awarded for resubmission will be \textbf{C-}.

\section*{Resits}
Resits \& reassessments \textbf{are not} applicable in \textbf{IN721: Mobile Application Development}. 

\section*{Instructions}
You have been provided starter code called \textbf{project-starter-code} located in the \textbf{code-resources} directory in the course materials repository. Carefully look at the given code \& understand what is happening. If you are unsure about anything, do not hesitate to contact the course lecturer. \\

You will need to submit an application \& documentation that meet the following requirements:

\subsection*{Functionality (Features) - Learning Outcomes 1, 2, 3 (40\%)}
\begin{itemize}
	\item Application \textbf{must} open without file structure modification in \textbf{Android Studio}.
	\item Application \textbf{must} run without code modification on a mobile device.
	\item Application \textbf{must} run on \textbf{API 28: Android 9.0 (Pie)}.
	\item \textbf{Independent Research:} Text translation support. If a country is multilingual (use of more than one language), choose one language. For example, Canada's main languages are English \& French. You would choose either English or French.
	      \begin{itemize}
	      	\item Use \textbf{Retrofit} \& the \textbf{Yandex Translate API} to translate text from one language to another. To use the \textbf{Yandex Translate API}, you will need an \textbf{API key}. A key will be privately given to you on \textbf{Microsoft Teams}. Ensure that the \textbf{API key} is not publicly exposed in your program files.
	      	\item Display some feedback while the text is being translated.
	      	\item Handle incorrectly formatted input fields. For example, an \textbf{EditText} is blank or empty.
	      \end{itemize}
	\item \textbf{Independent Research:} Text to speech support.
	      \begin{itemize}
	      	\item Handle incorrectly formatted input fields. For example, an \textbf{EditText} is blank or empty, or the country is not supported.
	      \end{itemize}
	\item \textbf{Independent Research:} Selection of three well-known phrases. For example, "No worries, mate, she'll be right" is well-known phrase in Australia.
	\item \textbf{Independent Research:} Register a new user on \textbf{Firebase}.
	\item Log into the application with an \textbf{email} \& \textbf{password} using \textbf{Firebase}.
	\item \textbf{Independent Research:} Logout of the application. The user \textbf{should} be navigated to the login screen.
	\item \textbf{Independent Research:} An interactive quiz for each country.
	      \begin{itemize}
	      	\item Quiz data \textbf{must} be fetched from a \textbf{GitHub Gist}.
	      	\item Quiz topics may include animals, culture, food, drink, geography \& sport.
	      	\item Each quiz \textbf{must} have at least three questions.
	      	\item Questions are multi-choice \& \textbf{must} have four answers.
	      	\item Each question \textbf{must} have an image.
	      	\item Display appropriate feedback in a \textbf{Toast} when a question is answered correctly or incorrectly. If an answer is incorrect, display the correct answer.
	      	\item A quiz \textbf{must} be completed within a \textbf{1.5 minute} time limit.
	      	\item At the end the quiz, store the user's \textbf{score} in a \textbf{Room Database} table.
	      	\item Display the user's highest \textbf{score} in a \textbf{TextView}.
	      \end{itemize}
	\item \textbf{Switch} which toggles between light \& dark mode.
	      \begin{itemize}
	      	\item The state (true or false) value of the \textbf{Switch} \textbf{must} be stored in a \textbf{DataStore}.
	      	\item The mode will be based off the state value of the \textbf{Switch}. For example, true equals dark mode \& false equals light mode.
	      	\item The mode must be persistence across \textbf{all} screens, i.e., if a user kills \& starts the application, the mode will be retrieved from a \textbf{DataStore}.
	      \end{itemize}
	\item \textbf{Google Map} displaying top-rated tourist attractions. 
	      \begin{itemize}
			\item Display \textbf{only} two tourist attractions per continent.
	      	\item Top-rated tourist attraction data \textbf{must} be fetched from a \textbf{GitHub Gist}.
	      	\item Each data object will represent a marker.
	      	\item The marker's information window \textbf{must} display the attraction's name \& city/town.
	      	\item \textbf{Independent Research:} If dark mode, set the map's style to a dark theme. If not, set the map's style to light theme.
	      	\begin{itemize}
				  \item \textbf{Resource:} \small\href{https://mapstyle.withgoogle.com}{https://mapstyle.withgoogle.com}
			  \end{itemize}
	      \end{itemize}
	\item Splash screen using a \textbf{Lottie} animation.
	\item Adaptive launcher icon which is displayed in a variety of shapes. The icon \textbf{must} be the same as the \textbf{Lottie} animation icon.
	\item Visually attractive UI with a coherent graphical theme \& style using \textbf{Material Design}.
	\item Application is published to \textbf{Google Play Store}.
	      \begin{itemize}
	      	\item To published to \textbf{Google Play Store}, you will need a \textbf{Google Play Console} account. The account's credentials will be privately given to you on \textbf{Microsoft Teams}. \textbf{Do not} disable any applications published on this account.
	      	\item When you create your application, name the package appropriately. For example, \\ \textbf{op.mobile.app.dev.$<$username$>$.travelling}. \textbf{Note:} replace \textbf{username} with your \textbf{Otago Polytechnic} username.
	      \end{itemize}
	\item Ability to download your application from \textbf{Google Play Store} on to a mobile device.
	\item UI tests which verify that the register, login \& logout is functioning correctly.
\end{itemize}

\subsection*{Code Elegance - Learning Outcomes 1, 3 (40\%)}
\begin{itemize}
	\item \textbf{Kotlin} \& \textbf{XML} files contain no magic numbers/strings. Store the values in the appropriate \textbf{XML} files. For example, numbers \textbf{must} be stored in an \textbf{integer.xml} or a \textbf{dimens.xml} file \& strings \textbf{must} be stored in a \textbf{strings.xml}.
	\item Idiomatic use of control flow, data structures \& in-built functions.
	\item Code adheres to \textbf{DRY}, \textbf{KISS} \& the \textbf{Model-View-ViewModel} architectural pattern.
	\item Efficient algorithmic approach.
	\item Commented code is documented using \textbf{KDoc}. The purpose of each class and function \textbf{must} be explained.
	\item \textbf{Kotlin} \& \textbf{XML} files are code formatted.
	\item No unused code \& resources.
\end{itemize}

\subsection*{Documentation \& Git/GitHub Usage - Learning Outcomes 2, 3 (20\%)}
\begin{itemize}
	\item Code commented is generated to \textbf{Markdown} using \textbf{Dokka}.
	\item Provide the following in your repository \textbf{README.md} file:
	      \begin{itemize}
	      	\item URL to your application's privacy policy.
	      	\item URL to commented code \textbf{Markdown} files.
	      	\item URL to your application on \textbf{Google Play Store}.
	      \end{itemize}			
	\item Commit messages \textbf{must}:
	\begin{itemize}
		\item Reflect the context of each functional requirement change. 
		\item Be formatted using the naming conventions outlined in the following:
		\begin{itemize}
			\item \textbf{Resource:} \small\href{https://dev.to/i5han3/git-commit-message-convention-that-you-can-follow-1709}{https://dev.to/i5han3/git-commit-message-convention-that-you-can-follow-1709}
		\end{itemize} 
	\end{itemize}
	\item You \textbf{must} commit at least \textbf{ten} times per week. By the end of this assessment, you should have at least \textbf{100} commits.
	\item \textbf{Do not} rewrite your \textbf{Git} history. It is important that the course lecturer can see how you worked on your assessment over time. 
\end{itemize}
\end{document}