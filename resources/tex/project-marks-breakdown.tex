% Author: Grayson Orr
% Course: ID721001: Mobile Application Development

\documentclass{article}
\author{}

\usepackage{graphicx}
\usepackage{wrapfig}
\usepackage{enumerate}
\usepackage{hyperref}
\usepackage[margin = 2.25cm]{geometry}
\usepackage[table]{xcolor}
\usepackage{fancyhdr}
\hypersetup{
  colorlinks = true,
  urlcolor = blue
}
\setlength\parindent{0pt}
\pagestyle{fancy}
\fancyhf{}
\rhead{College of Engineering, Construction and Living Sciences\\Bachelor of Information Technology}
\lfoot{Project Marks Breakdown\\Version 3, Semester One, 2022}
\rfoot{\thepage}
 
\begin{document}

\begin{figure}
	\centering
	\includegraphics[width=50mm]{../../resources/img/logo.png}
\end{figure}

\title{College of Engineering, Construction and Living Sciences\\Bachelor of Information Technology\\ID721001: Mobile Application Development\\Level 7, Credits 15\\\textbf{Project Marks Breakdown}}
\date{}
\maketitle

\subsection*{Functionality - Learning Outcomes 1, 2, 3 (40\%)}
\begin{itemize}
	\item Application \textbf{must} open without file structure modification in \textbf{Android Studio}.
	\item Application \textbf{must} run without code modification on a mobile device.
	\item Application \textbf{must} run on \textbf{API 28: Android 9.0 (Pie)}.
	\item All country data must be fetched using \textbf{Retrofit} from a \textbf{GitHub Gist}. You have been provided an example called \textbf{data.json} in the \textbf{course materials repository $>$ assessments}.
	\item Display the list of countries in a \textbf{RecyclerView}. Each item needs to have the country's name \& flag.
	\item \textbf{Independent Research:} Text translation support. If a country is multilingual (use of more than one language), choose one language. For example, Canada's main languages are English \& French. You would choose either English or French.
	      \begin{itemize}
		      \item Use \textbf{Retrofit} \& the \textbf{Yandex Translate API} to translate text from one language to another. To use the \textbf{Yandex Translate API}, you will need an \textbf{API key}. A key will be privately given to you on \textbf{Microsoft Teams}.
		      \item Display some feedback while the text is being translated.
		      \item Handle incorrectly formatted input fields. For example, an \textbf{EditText} is blank or empty.
	      \end{itemize}
	\item \textbf{Independent Research:} Text to speech support.
	      \begin{itemize}
		      \item Handle incorrectly formatted input fields. For example, an \textbf{EditText} is blank or empty, or the country is not supported.
	      \end{itemize}
	\item \textbf{Independent Research:} Selection of two well-known phrases. For example, "No worries, mate, she'll be right" is well-known phrase in Australia.
	\item \textbf{Independent Research:} Register a new user on \textbf{Firebase}.
	\item Log into the application with an \textbf{email} \& \textbf{password} using \textbf{Firebase}.
	\item \textbf{Independent Research:} Logout of the application. The user \textbf{should} be navigated to the login screen.
	\item \textbf{Independent Research:} An interactive quiz for each country.
	      \begin{itemize}
		      \item Quiz topics may include animals, culture, food, drink, geography \& sport.
		      \item Each quiz \textbf{must} have at least three questions.
		      \item Questions are multi-choice \& \textbf{must} have four answers.
		      \item Display appropriate feedback in a \textbf{Toast} when a question is answered correctly or incorrectly. If an answer is incorrect, display the correct answer.
		      \item At the end the quiz, store the user's \textbf{id}, \textbf{name} \& \textbf{score}, \& the country's \textbf{id} in a \textbf{Room Database} table.
		      \item Display all scores in a \textbf{RecylerView}. Each item needs to have the user's \textbf{name} \& \textbf{score}.
		      \item Delete \textbf{all} scores. You need to prompt the user to confirm deletion.
	      \end{itemize}
	\item \textbf{Switch} which toggles between light \& dark mode.
	      \begin{itemize}
		      \item The state (true or false) value of the \textbf{Switch} \textbf{must} be stored in a \textbf{DataStore}.
		      \item The mode will be based off the state value of the \textbf{Switch}. For example, true equals dark mode \& false equals light mode.
		      \item The mode must be persistence across \textbf{all} screens, i.e., if a user kills \& starts the application, the mode will be retrieved from a \textbf{DataStore}.
	      \end{itemize}
	\item \textbf{Independent Research:} \textbf{Google Map} displaying top-rated tourist attractions.
	      \begin{itemize}
		      \item Display two tourist attractions per continent.
		      \item Each data object will represent a marker.
		      \item The marker's information window \textbf{must} display the attraction's name \& city/town.
		      \item If dark mode, set the map's style to a dark theme. If not, set the map's style to light theme.
		            \begin{itemize}
			            \item \textbf{Resource:} \small\href{https://mapstyle.withgoogle.com}{https://mapstyle.withgoogle.com}
		            \end{itemize}
		      \item To use \textbf{Google Map}, you will need an \textbf{API key}. A key will be privately given to you on \textbf{Microsoft Teams}.
	      \end{itemize}
	\item Splash screen using a \textbf{Lottie} animation.
	\item Like \& favourite a country. Store the user's \textbf{id}, country's \textbf{id} \& \textbf{type}, i.e., like or favourite in a \textbf{Room Database} table. 
	\item \textbf{Independent Research:} Adaptive launcher icon which is displayed in a variety of shapes. The icon \textbf{must} be the same as the \textbf{Lottie} animation icon.
	\item Visually attractive UI with a coherent graphical theme \& style using \textbf{Material Design}.
	\item Application is published to \textbf{Google Play Store}.
	      \begin{itemize}
		      \item To published to \textbf{Google Play Store}, you will need a \textbf{Google Play Console} account. The account's credentials will be privately given to you on \textbf{Microsoft Teams}. \textbf{Do not} disable any applications published on this account.
		      \item When you create your application, name the package appropriately. For example, \\ \textbf{op.mobile.app.dev.$<$username$>$.travelling}. \textbf{Note:} replace \textbf{username} with your \textbf{Otago Polytechnic} username.
	      \end{itemize}
	\item Ability to download your application from \textbf{Google Play Store} on to a mobile device.
	\item UI tests which verify that the register, login \& logout is functioning correctly.
\end{itemize}

\subsection*{Code Elegance - Learning Outcomes 1, 3 (45\%)}
\begin{itemize}
	\item \textbf{Kotlin} \& \textbf{XML} files contain no magic numbers/strings. Store the values in the appropriate \textbf{XML} files. For example, numbers \textbf{must} be stored in an \textbf{integer.xml} or a \textbf{dimens.xml} file \& strings \textbf{must} be stored in a \textbf{strings.xml}.
	\item Idiomatic use of control flow, data structures \& in-built functions.
	\item Code adheres to \textbf{DRY}, \textbf{KISS} \& the \textbf{Model-View-ViewModel} architectural pattern.
	\item Efficient algorithmic approach.
	\item Commented code is documented using \textbf{KDoc}. The purpose of each class and function \textbf{must} be explained.
	\item API keys are stored \& retrieved from \textbf{local.properties}.
	\item \textbf{Kotlin} \& \textbf{XML} files are code formatted.
	\item No unused code \& resources.
\end{itemize}

\subsection*{Documentation \& Git/GitHub Usage - Learning Outcomes 2, 3 (15\%)}
\begin{itemize}
	\item Provide the following in your repository \textbf{README.md} file:
	      \begin{itemize}
		      \item URL to your application's privacy policy.
		      \item URL to commented code generated to \textbf{Markdown} using \textbf{Dokka}.
		      \item URL to your application on \textbf{Google Play Store}.
	      \end{itemize}
	\item Commit messages \textbf{must}:
	      \begin{itemize}
		      \item Reflect the context of each functional requirement change.
		      \item Be formatted using the naming conventions outlined in the following:
		            \begin{itemize}
			            \item \textbf{Resource:} \small\href{https://dev.to/i5han3/git-commit-message-convention-that-you-can-follow-1709}{https://dev.to/i5han3/git-commit-message-convention-that-you-can-follow-1709}
		            \end{itemize}
	      \end{itemize}
\end{itemize}
\end{document}