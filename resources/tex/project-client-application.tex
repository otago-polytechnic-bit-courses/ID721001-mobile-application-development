% Author: Grayson Orr
% Course: ID721001: Mobile Application Development

\documentclass{article}
\author{}

\usepackage{graphicx}
\usepackage{wrapfig}
\usepackage{enumerate}
\usepackage{hyperref}
\usepackage[margin = 2.25cm]{geometry}
\usepackage[table]{xcolor}
\usepackage{fancyhdr}
\hypersetup{
  colorlinks = true,
  urlcolor = blue
}
\setlength\parindent{0pt}
\pagestyle{fancy}
\fancyhf{}
\rhead{College of Engineering, Construction and Living Sciences\\Bachelor of Information Technology}
\lfoot{Project: Client Application \\Version 1, Semester Two, 2023}
\rfoot{\thepage}
 
\begin{document}

\begin{figure}
	\centering
	\includegraphics[width=50mm]{../img/logo.png}
\end{figure}

\title{College of Engineering, Construction and Living Sciences\\Bachelor of Information Technology\\ID721001: Mobile Application Development\\Level 7, Credits 15\\\textbf{Project: Client Application }}
\date{}
\maketitle

\section*{Assessment Overview}
In this \textbf{individual} assessment, you will develop a mobile application using \textbf{React Native} and \textbf{Expo}, and publish it to \textbf{Google Play Store} or \textbf{Apple App Store}. In addition, marks will be allocated for code elegance, documentation and \textbf{Git} usage.

\section*{Learning Outcomes}
At the successful completion of this course, learners will be able to:
\begin{enumerate}
	\item Implement and publish complete, non-trivial, industry-standard mobile applications following sound architectural and code-quality standards.
	\item Identify relevant use cases for a mobile computing scenario and incorporate them into an effective user experience design.
	\item Follow industry standard software engineering practice in the design of mobile applications.
\end{enumerate}

\section*{Assessments}
\renewcommand{\arraystretch}{1.5}
\begin{tabular}{|c|c|c|c|}
	\hline
	\textbf{Assessment} & \textbf{Weight} & \textbf{Due Date}    & \textbf{Learning Outcomes} \\ \hline
	Project: Client Application          & 60\%            & 10-11-2023 (Friday at 4.59 PM)  & 1, 2, 3                    \\ \hline
	Practical: Skills-Based           & 20\%            & 22-09-2023 (Friday at 4.59 PM)  & 1, 2, 3                    \\ \hline
	Presentation: Client Application       & 20\%            & 10-11-2023 (Friday at 4.59 PM) & 2, 3                       \\ \hline
\end{tabular} 

\section*{Conditions of Assessment}
You will complete majority of this assessment during your learner-managed time. However, there will be time during class to discuss the requirements and your progress on this assessment. This assessment will need to be completed by \textbf{Friday, 10 November 2023} at \textbf{4.59 PM}.

\section*{Pass Criteria}
This assessment is criterion-referenced (CRA) with a cumulative pass mark of \textbf{50\%} over all assessments in \textbf{ID721001: Mobile Application Development}.

\section*{Authenticity}
All parts of your submitted assessment \textbf{must} be completely your work. Do your best to complete this assessment without using an \textbf{AI generative tool}. You need to demonstrate to the course lecturer that you can meet the learning outcome for this assessment. \\
 
 However, if you get stuck, you can use an \textbf{AI generative tool} to help you get unstuck, permitting you to acknowledge that you have used it. In the assessment's repository \textbf{README.md} file, please include what prompt(s) you provided to the \textbf{AI generative tool} and how you used the response(s) to help you with your work. It also applies to code snippets retrieved from \textbf{StackOverflow} and \textbf{GitHub}. \\
 
 Failure to do this may result in a mark of \textbf{zero} for this assessment.

\section*{Policy on Submissions, Extensions, Resubmissions and Resits}
The school's process concerning submissions, extensions, resubmissions and resits complies with \textbf{Otago Polytechnic} policies. Learners can view policies on the \textbf{Otago Polytechnic} website located at \href{https://www.op.ac.nz/about-us/governance-and-management/policies}{https://www.op.ac.nz/about-us/governance-and-management/policies}.

\section*{Submission}
You \textbf{must} submit all project files via \textbf{GitHub Classroom}. Here is the URL to the repository you will use for your submission – \href{https://classroom.github.com/a/yitUo0I6}{https://classroom.github.com/a/yitUo0I6}.  Create a \textbf{.gitignore} and add the ignored files in this resource - \href{https://raw.githubusercontent.com/github/gitignore/main/Node.gitignore}{https://raw.githubusercontent.com/github/gitignore/main/Node.gitignore}. The latest project files in the \textbf{master} or \textbf{main} branch will be used to mark against the \textbf{Functionality} criterion. Please test before you submit. Partial marks \textbf{will not} be given for incomplete functionality. Late submissions will incur a \textbf{10\% penalty per day}, rolling over at \textbf{5:00 PM}.

\section*{Extensions}
Familiarise yourself with the assessment due date. If you need an extension, contact the course lecturer before the due date. If you require more than a week's extension, a medical certificate or support letter from your manager may be needed.

\section*{Resubmissions}
Learners may be requested to resubmit an assessment following a rework of part/s of the original assessment. Resubmissions are to be completed within a negotiable short time frame and usually \textbf{must} be completed within the timing of the course to which the assessment relates. Resubmissions will be available to learners who have made a genuine attempt at the first assessment opportunity and achieved a \textbf{D grade (40-49\%)}. The maximum grade awarded for resubmission will be \textbf{C-}.

\section*{Resits}
Resits and reassessments \textbf{are not} applicable in \textbf{ID721001: Mobile Application Development}.

\section*{Instructions}
You will need to submit a mobile application and documentation that meet the following requirements:

\subsection*{Functionality - Learning Outcomes 1, 2, 3 (50\%)}
\begin{itemize}
	\item The mobile application needs to run without code or file structure modification in \textbf{Visual Studio Code}.
	\item Adhere to the functionality requirements outlined by the client.
	\item Usable on a variety of mobile devices, i.e., devices with different screen sizes.
	\item Free of bugs that significantly effect the usability.
	\item The mobile application is published to \textbf{Google Play Store} or \textbf{Apple App Store}.
	      \begin{itemize}
		      \item To published to \textbf{Google Play Store} or \textbf{Apple App Store}, you will need an account. The account's credentials will be privately given to you on \textbf{Microsoft Teams}. \textbf{Do not} disable any applications published on this account.
	      \end{itemize}
	\item Ability to download the mobile application from \textbf{Google Play Store} or \textbf{Apple App Store} on to a variety of mobile devices.
\end{itemize}

\subsection*{Code Elegance - Learning Outcomes 1, 3 (40\%)}
\begin{itemize}
	\item A \textbf{Node.js} \textbf{.gitignore} file is used.
	\item If applicable, a \textbf{.env} and \textbf{.env.example} file is used.
  \item Appropriate naming of files, variables, functions and components.
	\item Idiomatic use of control flow, data structures and in-built functions.
  \item Efficient algorithmic approach.
  \item Sufficient modularity.
  \item Each \textbf{component} file \textbf{must} have a \textbf{JSDoc} header comment located immediately before the \textbf{import} statements.
\item In-line comments where required. It should be for code that needs further explanation.
  \item Code is formatted.	
\item No dead or unused code. 
\end{itemize}

\subsection*{Documentation and Git/GitHub Usage - Learning Outcomes 2, 3 (10\%)}
\begin{itemize}
	\item \textbf{GitHub} project board to help you organise and prioritise your work. 
    \item Provide the following in your repository \textbf{README.md} file:
    \begin{itemize} 
	  \item Link to the mobile game on \textbf{Google Play Store} or \textbf{Apple App Store}.
      \item At least five initial functionality requirements.
      \item Wireframes of the mobile application's screens. The wireframes can be either hand-drawn or created using a digital tool.
	  \item How do you setup the environment, i.e., after the repository is cloned?
      \item If applicable, known bugs.
    \end{itemize}
    \item Use of \textbf{Markdown}, i.e., headings, bold text, code blocks, etc.
    \item Correct spelling and grammar.
    \item Your \textbf{Git commit messages} should:
    \begin{itemize}
      \item Reflect the context of each functional requirement change.
      \item Be formatted using an appropriate naming convention style.
    \end{itemize}
\end{itemize} 

\subsection*{Additional Information}
\begin{itemize}
	\item \textbf{Do not} rewrite your \textbf{Git} history. It is important that the course lecturer can see how you worked on your assessment over time.
	\item You need to provide the five initial functionality requirements and wireframes to the course lecturer before you begin development.
\end{itemize}

\end{document}