% Author: Grayson Orr
% Course: ID721001: Mobile Application Development

\documentclass{article}
\author{}

\usepackage{graphicx}
\usepackage{wrapfig}
\usepackage{enumerate}
\usepackage{hyperref}
\usepackage[margin = 2.25cm]{geometry}
\usepackage[table]{xcolor}
\usepackage{fancyhdr}
\hypersetup{
  colorlinks = true,
  urlcolor = blue
}
\setlength\parindent{0pt}
\pagestyle{fancy}
\fancyhf{}
\rhead{College of Engineering, Construction \& Living Sciences\\Bachelor of Information Technology}
\lfoot{Project\\Version 2, Semester Two, 2023}
\rfoot{\thepage}
 
\begin{document}

\begin{figure}
	\centering
	\includegraphics[width=50mm]{../img/logo.png}
\end{figure}

\title{College of Engineering, Construction \& Living Sciences\\Bachelor of Information Technology\\ID721001: Mobile Application Development\\Level 7, Credits 15\\\textbf{Project}}
\date{}
\maketitle

\section*{Assessment Overview}
In this \textbf{individual} assessment, you will develop a mobile game using \textbf{Unity} \& publish it to \textbf{Google Play Store}. Also, you will provide documentation that addresses several aspects of game development. In addition to the mobile game \& documentation, you will conduct an evaluative conversation. 

\section*{Learning Outcomes}
At the successful completion of this course, learners will be able to:
\begin{enumerate}
	\item Implement \& publish complete, non-trivial, industry-standard mobile applications following sound architectural \& code-quality standards.
	\item Identify relevant use cases for a mobile computing scenario \& incorporate them into an effective user experience design.
	\item Follow industry standard software engineering practice in the design of mobile applications.
\end{enumerate}

\section*{Assessment Table}
\renewcommand{\arraystretch}{1.5}
\begin{tabular}{|l|l|l|l|l|}
	\hline
	\vtop{\hbox{\strut \textbf{Assessment}}\hbox{\strut \textbf{Activity}}} & \textbf{Weighting} & \vtop{\hbox{\strut \textbf{Learning}}\hbox{\strut \textbf{Outcomes}}} & \vtop{\hbox{\strut \textbf{Assessment}}\hbox{\strut \textbf{Grading Scheme}}} & \vtop{\hbox{\strut \textbf{Completion}}\hbox{\strut \textbf{Requirements}}} \\

	\hline
	\small Project                                                          & \small 100\%        & \small 1, 2, 3                                                        & \small CRA                                                                    & \small Cumulative                                                           \\ \hline
\end{tabular}

\section*{Conditions of Assessment}
You will complete majority of this assessment during your learner-managed time. However, there will be time during class to discuss the requirements and your progress on this assessment. This assessment will need to be completed by \textbf{Friday, 10 November 2023} at \textbf{4.59 PM}.

\section*{Pass Criteria}
This assessment is criterion-referenced (CRA) with a cumulative pass mark of \textbf{50\%} over all assessments in \textbf{ID721001: Mobile Application Development}.

\section*{Authenticity}
All parts of your submitted assessment \textbf{must} be completely your work. Do your best to complete this assessment without using an \textbf{AI generative tool}. You need to demonstrate to the course lecturer that you can meet the learning outcome for this assessment. \\
 
 However, if you get stuck, you can use an \textbf{AI generative tool} to help you get unstuck, permitting you to acknowledge that you have used it. In the assessment's repository \textbf{README.md} file, please include what prompt(s) you provided to the \textbf{AI generative tool} and how you used the response(s) to help you with your work. It also applies to code snippets retrieved from \textbf{StackOverflow} and \textbf{GitHub}. \\
 
 Failure to do this may result in a mark of \textbf{zero} for this assessment.

\section*{Policy on Submissions, Extensions, Resubmissions \& Resits}
The school's process concerning submissions, extensions, resubmissions \& resits complies with \textbf{Otago Polytechnic} policies. Learners can view policies on the \textbf{Otago Polytechnic} website located at \href{https://www.op.ac.nz/about-us/governance-and-management/policies}{https://www.op.ac.nz/about-us/governance-and-management/policies}.

\section*{Submission}
You \textbf{must} submit all program files via \textbf{GitHub}. The latest program files in the \textbf{master} or \textbf{main} branch will be used to mark against the \textbf{Functionality} criterion. Please test your \textbf{master} or \textbf{main} branch application before you submit. Partial marks \textbf{will not} be given for incomplete functionality. Late submissions will incur a \textbf{10\% penalty per day}, rolling over at \textbf{5:00 PM}.

\section*{Extensions}
Familiarise yourself with the assessment due date. If you need an extension, contact the course lecturer before the due date. If you require more than a week's extension, a medical certificate or support letter from your manager may be needed.

\section*{Resubmissions}
Learners may be requested to resubmit an assessment following a rework of part/s of the original assessment. Resubmissions are to be completed within a negotiable short time frame \& usually \textbf{must} be completed within the timing of the course to which the assessment relates. Resubmissions will be available to learners who have made a genuine attempt at the first assessment opportunity \& achieved a \textbf{D grade (40-49\%)}. The maximum grade awarded for resubmission will be \textbf{C-}.

\section*{Resits}
Resits \& reassessments \textbf{are not} applicable in \textbf{ID721001: Mobile Application Development}.

\section*{Instructions}
You will need to submit a mobile game \& documentation that meet the following requirements:

\subsection*{Functionality - Learning Outcomes 1, 2, 3 (60\%)}
\begin{itemize}
	\item Mobile game \textbf{must} open without code or file structure modification in \textbf{Unity}.
	\item Playable on a variety of mobile devices, i.e., devices with different screen sizes.
	\item Free of bugs that significantly effect the playability.
	\item Mobile game is published to \textbf{Google Play Store}.
	      \begin{itemize}
		      \item To published to \textbf{Google Play Store}, you will need a \textbf{Google Play Console} account. The account's credentials will be privately given to you on \textbf{Microsoft Teams}. \textbf{Do not} disable any applications published on this account.
	      \end{itemize}
	\item Ability to download your Mobile game from \textbf{Google Play Store} on to a variety of mobile devices.
\end{itemize}

\subsection*{Documentation - Learning Outcomes 2, 3 (20\%)}
\begin{itemize}
	\item Provide the following in your repository \textbf{README.md} file:
	\begin{itemize}
		\item Game overview:
		\begin{itemize}
			\item What is the game's concept and genre?
		\end{itemize}
		\item Gameplay mechanics:
		\begin{itemize}
			\item What is the game's core gameplay loop?
			\item How do the players interact with the game and control the characters or objects?
			\item 
		\end{itemize}
		\item Art and visuals:
		\begin{itemize}
			\item What art style is used?
		\end{itemize}
		\item Audio and sound:
		\begin{itemize}
			\item What audio and sound effects are used?
		\end{itemize}
		\item If applicable, known bugs.
	\end{itemize}
\end{itemize}

\subsection*{Evaluative Conversation - Learning Outcomes 2, 3 (20\%)}
\begin{itemize}
	\item You must reflect on the following:
	      \begin{itemize}
			\item What is your mobile game?
			\item What considerations did you make when planning your mobile game?
			\item How did you effectively use of your chosen game engine?
			\item What are some areas for improvement?
			\item If you were to continue with the mobile game, what are the next steps?
	      \end{itemize}
\end{itemize}

\subsection*{Additional Information}
\begin{itemize}
	\item \textbf{Do not} rewrite your \textbf{Git} history. It is important that the course lecturer can see how you worked on your assessment over time.
\end{itemize}

\end{document}