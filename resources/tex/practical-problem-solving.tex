% Author: Grayson Orr
% Course: ID721: Mobile Application Development

\documentclass{article}
\author{}

\usepackage{graphicx}
\usepackage{wrapfig}
\usepackage{enumerate}
\usepackage{hyperref}
\usepackage[margin = 2.25cm]{geometry}
\usepackage[table]{xcolor}
\usepackage{fancyhdr}
\hypersetup{
  colorlinks = true,
  urlcolor = blue
}
\setlength\parindent{0pt}
\pagestyle{fancy}
\fancyhf{}
\rhead{College of Engineering, Construction \& Living Sciences\\Bachelor of Information Technology}
\lfoot{Practical: Problem-Solving\\Version 2, Semester Two, 2022}
\rfoot{\thepage}

\begin{document}

\begin{figure}
  \centering
  \includegraphics[width=50mm]{../../resources/img/logo.png}
\end{figure}

\title{College of Engineering, Construction \& Living Sciences\\Bachelor of Information Technology\\ID721001: Mobile Application Development\\Level 7, Credits 15\\\textbf{Practical: Problem-Solving}}
\date{}
\maketitle

\section*{Assessment Overview}
In this \textbf{individual} assessment, you will solve \textbf{20 coding problems} using \textbf{Kotlin} \& \textbf{IntelliJ IDEA}.

\section*{Learning Outcomes}
At the successful completion of this course, learners will be able to:
\begin{enumerate}
  \item Implement \& publish complete, non-trivial, industry-standard mobile applications following sound architectural \& code-quality standards.
  \item Identify relevant use cases for a mobile computing scenario \& incorporate them into an effective user experience design.
  \item Follow industry standard software engineering practice in the design of mobile applications.
\end{enumerate}

\section*{Assessments}
\renewcommand{\arraystretch}{1.5}
\begin{tabular}{|c|c|c|c|}
	\hline
	\textbf{Assessment} & \textbf{Weight} & \textbf{Due Date}    & \textbf{Learning Outcomes} \\ \hline
	Project: Travelling Application             & 60\%            & 28-10-2022 (Friday at 4.59 PM)  & 1, 2, 3                    \\ \hline
	Practical: Problem-Solving          & 20\%            & 26-08-2022 (Friday at 4.59 PM) & 3                    \\ \hline
	Presentation: Advanced Android Topic        & 20\%            & 16-11-2022 (Wednesday at 4.59 PM) & 3                       \\ \hline
\end{tabular}

\section*{Conditions of Assessment}
You will complete this assessment during your learner-managed time. However, there will be time during class to discuss the requirements \& your progress on this assessment. This assessment will need to be completed by \textbf{Friday, 26 August 2022} at \textbf{4.59 PM}.

\section*{Pass Criteria}
This assessment is criterion-referenced (CRA) with a cumulative pass mark of \textbf{50\%} over all assessments in \textbf{ID721001: Mobile Application Development}.

\section*{Authenticity}
All parts of your submitted assessment \textbf{must} be completely your work. If you use code snippets from \textbf{GitHub}, \textbf{StackOverflow} or other online resources, you \textbf{must} reference it appropriately using \textbf{APA 7th edition}. Provide your references in the \textbf{README.md} file in your repository. Failure to do this will result in a mark of \textbf{zero} for this assessment.

\section*{Policy on Submissions, Extensions, Resubmissions \& Resits}
The school's process concerning submissions, extensions, resubmissions \& resits complies with \textbf{Otago Polytechnic} policies. Learners can view policies on the \textbf{Otago Polytechnic} website located at \href{https://www.op.ac.nz/about-us/governance-and-management/policies}{https://www.op.ac.nz/about-us/governance-and-management/policies}.

\section*{Submission}
You \textbf{must} submit all project files via \textbf{GitHub Classroom}. Here is the URL to the repository you will use for your submission – \href{https://classroom.github.com/a/rWCfXF\_l}{https://classroom.github.com/a/rWCfXF\_l}.  Create a \textbf{.gitignore} \& add the ignored files in this resource - \href{https://raw.githubusercontent.com/github/gitignore/main/Java.gitignore}{https://raw.githubusercontent.com/github/gitignore/main/Java.gitignore}. The latest project files in the \textbf{master} or \textbf{main} branch will marked. Please test before you submit. Partial marks \textbf{will not} be given for incomplete functionality. Late submissions will incur a \textbf{10\% penalty per day}, rolling over at \textbf{5:00 PM}.

\section*{Extensions}
Familiarise yourself with the assessment due date. If you need an extension, contact the course lecturer before the due date. If you require more than a week's extension, a medical certificate or support letter from your manager may be needed.

\section*{Resubmissions}
Learners may be requested to resubmit an assessment following a rework of part/s of the original assessment. Resubmissions are to be completed within a negotiable short time frame \& usually \textbf{must} be completed within the timing of the course to which the assessment relates. Resubmissions will be available to learners who have made a genuine attempt at the first assessment opportunity \& achieved a \textbf{D grade (40-49\%)}. The maximum grade awarded for resubmission will be \textbf{C-}.

\section*{Resits}
Resits \& reassessments \textbf{are not} applicable in \textbf{ID721001: Mobile Application Development}.

\section*{Instructions - Learning Outcomes 3}
Create a file for each problem.

\subsection*{Problem 1 (0.5\%):}
Calculate the average of the given \textbf{double array} \& display the expected output.

\begin{verbatim}
  fun main() {
      val nums = doubleArrayOf(45.3, 67.5, -45.6, 20.34, -33.0, 45.6)

      // Write your solution here

      // Expected output:
      // Average: 16.69 
  }
\end{verbatim}

\subsection*{Problem 2 (1\%):}
Write a function called \textbf{fizzBuzz} which accepts an \textbf{Int} parameter called \textbf{num}. If \textbf{num} is a multiple of three, return \textbf{Fizz}, if \textbf{num} is a multiple of five, return \textbf{Buzz} \& if \textbf{num} is a multiple of three \& five, return \textbf{FizzBuzz}. Call the \textbf{fizzBuzz} function in the \textbf{main} function to display the expected output.

\begin{verbatim}
  // Write your fizzBuzz function here
  
  fun main() {
      for (i in 1..15 step 2) {
        // Write your solution here
      }

      // Expected output:
      // 1
      // Fizz
      // Buzz
      // 7
      // Fizz
      // 11
      // 13
      // FizzBuzz
  }
\end{verbatim}

\subsection*{Problem 3 (0.5\%):} You have been given two \textbf{mutable lists} containing the lecturer's favourite programming languages. Use the following hints to display the expected output:
\begin{itemize}
  \item Add a specified element to the end of a list.
  \item Add all elements of a specified collection to the end of a list.
  \item If present, remove a specified element from a collection.
  \item Capitalise the element in the 3rd index.
\end{itemize}

\begin{verbatim}
  fun main() {
      val progLangsOne: MutableList<String> = mutableListOf("C#", "Java", "Kotlin", "Rust")
      val progLangsTwo: MutableList<String> = mutableListOf("C++", "Go", "Swift", "TypeScript")
    
      // Write your solution here
    
      // Expected output:
      // [C#, Java, Kotlin, RUST, Prolog, C++, Swift]
  }
\end{verbatim}

\subsection*{Problem 4 (0.5\%):} You have been given a \textbf{mutable map} containing three soft drinks \& their prices. Use the following hints and \textbf{Kotlin} aggregate operations to display the expected output:
\begin{itemize}
  \item Change the price of Coca-Cola to 5.50.
  \item Change the price of Spite to 0.10.
  \item Calculate the total price of all soft drinks.
\end{itemize}

\begin{verbatim}
  fun main() {
      val softDrinks: MutableMap<String, Double> 
          = mutableMapOf("Coca-Cola" to 2.00, "Fanta" to 0.90, "Sprite" to 1.10)

      // Write your solution here
			
      // Expected output:
      // Total price: $6.50
  }
\end{verbatim}

\subsection*{Problem 5 (0.5\%):} You have been given two \textbf{mutable sets} containing two lecturer's course codes. Use the following hints to display the expected output:
\begin{itemize}
  \item Return a set containing all elements that are contained by both collections.
  \item Return a set containing all distinct elements from both collections.
\end{itemize}

\begin{verbatim}
fun main() {
    val courseCodesOne: MutableSet<String> = mutableSetOf("ID607", "ID721", "ID728", "ID732")
    val courseCodesTwo: MutableSet<String> = mutableSetOf("ID512", "ID607", "ID728", "ID732")
    
    // Write your solution here
    
    // Expected output:
    // [ID607, ID728, ID732]
    // [ID607, ID721, ID728, ID732, ID512] 
}
\end{verbatim}

\subsection*{Problem 6 (2\%):}
You have been given a 5x5 grid or a \textbf{2D array} of zeros. Use the appropriate construct(s)/range(s) to access the items in the grid, i.e., zeros \& replace them with Xs.

\begin{verbatim}
  fun main() {
      var seating = arrayOf<Array<Any>>()
      for (i in 0..4) {
          var seat = arrayOf<Any>()
          for (j in 0..4) {
              seat += 0
          }
          seating += seat
      }

      // Write your solution here

      for (seat in seating) {
          for (value in seat) {
              print("$value ")
          }
          println()
      }

      // Expected output:
      // 0 0 0 0 X 
      // 0 0 0 0 0 
      // X X X 0 X 
      // 0 0 0 0 0 
      // 0 0 0 0 X
  }
\end{verbatim}

\subsection*{Problem 7 (1\%):}
In the expected output below, the staircase is of size three. Its base \& height are both equal to \textbf{numOfSteps}. Also, it is drawn using the hash symbol. Write the logic in the \textbf{generateSteps} function in order to display the expected output.

\begin{verbatim}
  fun generateSteps(numOfSteps: Int): MutableList<String> {
      val stepSeq: MutableList<String> = mutableListOf()
      
      // Write your solution here
      
      return stepSeq  
  }

  fun main() {
      for (step in generateSteps(4)) {
          // Expected output:
          println(step) // #  
                        // ## 
                        // ###
                        // ####
      }
  }
\end{verbatim}

\subsection*{Problem 8 (0.5\%):}
You have been given a function called \textbf{defangAddress} which accepts a \textbf{String} parameter called \textbf{address}. This function returns a defanged version of \textbf{address}. A defanged address replaces every period \textbf{"."} with \textbf{"[.]"}. Write the logic in the \textbf{defangAddress} function in order to display the expected output.

\begin{verbatim}
  fun defangAddress(address: String): String {
      var defangedAddr = ""
      
      // Write your solution here
      
      return defangedAddr
  }

  fun main() {
      // Expected output:
      println(defangAddress("255.100.50.0")) // 255[.]100[.]50[.]0
  }
\end{verbatim}

\subsection*{Problem 9 (1\%):}
You have been given an incomplete function called \textbf{isPerfectNumber} which accepts an \textbf{Int} parameter called \textbf{num}. If \textbf{num} is a perfect number, return \textbf{true}, otherwise return \textbf{false}. A perfect num is a positive integer that is equal to the sum of its positive divisors excluding the number itself.

\begin{verbatim}
  // Example 1
  Input: num = 6
  Output: true

  // Example 2
  Input: num = 2
  Output: false
\end{verbatim}

\begin{verbatim}
  fun isPerfectNumber(num: Int): Boolean {
      // Write your solution here
  }

  fun main() {
      // Expected output:
      println(isPerfectNumber(5)) // false
      println(isPerfectNumber(6)) // true
  }
\end{verbatim}

\subsection*{Problem 10 (0.5\%):}
You have been given an incomplete function called \textbf{removeDuplicates} which accepts an \textbf{IntArray} parameter called \textbf{nums}. Given a sorted \textbf{integer array,} remove the duplicates such that each element occurs only once \& return the new length of the \textbf{array}.

\begin{verbatim}
  fun removeDuplicates(nums: IntArray): Int {
      // Write your solution here  
  }

  fun main() {
      // Expected output:
      println(removeDuplicates(intArrayOf(0, 0, 1, 1, 2, 2, 3, 3, 4))) // 5
  }
\end{verbatim}

\subsection*{Problem 11 (2\%):}
Write two classes called \textbf{SoftwareDeveloper} \& \textbf{Manager} which inherit from the given \textbf{Employee} class. The \textbf{SoftwareDeveloper} class has one additional class property called \textbf{favProgLang} of type \textbf{String}. The \textbf{Manager} class also has one additional class property called \textbf{employees} of type \textbf{MutableList$<$Employee$>$} \& three functions which add, remove \& display all managed employees. \\

Use the three \textbf{SoftwareDeveloper} objects \& \textbf{Manager} object in the \textbf{main} function to display the expected output.

\begin{verbatim}
  open class Employee(var id: Int, val firstName: String, val lastName: String, val salary: Int) {
      override fun toString() = "${firstName} ${lastName}"
  }

  // Write your SoftwareDeveloper class here

  // Write your Manager class here

  fun main() {
      val sftDevOne = SoftwareDeveloper(1, "Bert", "Watts", 100000, "Cobol")
      val sftDevTwo = SoftwareDeveloper(2, "Sara", "Cain", 75000, "Perl")
      val sftDevThree = SoftwareDeveloper(3, "Samantha", "Baker", 75000, "PHP")
      val manager = Manager(4, "Owen", "James", 150000, mutableListOf(sftDevOne, sftDevTwo))

      // Write your solution here

      // Expected output:
      // Sara Cain
      // Samantha Baker
  }
\end{verbatim}

\subsection*{Problem 12 (1\%):}
You have been given a class called \textbf{Stack} of type \textbf{String}. Use the \textbf{Stack} object in the \textbf{main} function to display the expected output.

\begin{verbatim}
  class Stack<String>() {
      private val els = mutableListOf<String>()
      fun push(el: String) = els.add(el)
      fun peek(): String = els.last()
      fun pop(): String = els.removeAt(els.size - 1)
      fun isEmpty() = els.isEmpty()
      fun size() = els.size
      override fun toString() = "Stack[${els.joinToString()}]"
  }

  fun main() {
      val stack: Stack<String> = Stack()
      stack.push("Express")
      stack.push("Laravel")
      stack.push("Ruby on Rails")
      stack.push("Spring")

      // Write your solution here

      // Expected output:
      // Stack[Express, Laravel, Ruby on Rails]
      // Ruby on Rails is at the top of the stack
      // There are 3 item(s) in the stack
  }
\end{verbatim}

\subsection*{Problem 13 (1\%):}
You have been given a class called \textbf{Stack} of type \textbf{String}. Use the \textbf{Stack} object in the \textbf{main} function \& the \textbf{readLine} function to reverse the user's input.

\begin{verbatim}
  class Stack<String>() {
      private val els = mutableListOf<String>()
      fun push(el: String) = els.add(el)
      fun peek(): String = els.last()
      fun pop(): String = els.removeAt(els.size - 1)
      fun isEmpty() = els.isEmpty()
      fun size() = els.size
      override fun toString() = "Stack[${els.joinToString()}]"
  }

  fun main() {
      val stack: Stack<String> = Stack()

      // Write your solution here

      // Expected output:
      // Enter some text: John Doe
      // eoD nhoJ
  }
\end{verbatim}

\subsection*{Problem 14 (1\%):}
You have been given a class called \textbf{Stack} of type \textbf{Int}. Use the \textbf{Stack} object in the \textbf{main} function \& the \textbf{readLine} function to convert the user's input into binary.

\begin{verbatim}
  class Stack<Int>() {
      private val els = mutableListOf<Int>()
      fun push(el: Int) = els.add(el)
      fun peek(): Int = els.last()
      fun pop(): Int = els.removeAt(els.size - 1)
      fun isEmpty() = els.isEmpty()
      fun size() = els.size
      override fun toString() = "Stack[${els.joinToString()}]"
  }

  fun main() {
      val stack: Stack<Int> = Stack()

      // Write your solution here

      // Expected output:
      // Enter a number: 50
      // 110010
  }
\end{verbatim}

\subsection*{Problem 15 (1\%):}
You have been given a class called \textbf{Stack} of type \textbf{Char} \& an incomplete function called \textbf{isBalanced} which accepts a \textbf{String} parameter called \textbf{sequence}. Given a \textbf{sequence} containing only parentheses, curly brackets \& square brackets, determine if \textbf{sequence} is valid.

\begin{verbatim}
  class Stack<Char>() {
      private val els = mutableListOf<Char>()
      fun push(el: Char) = els.add(el)
      fun peek(): Char = els.last()
      fun pop(): Char = els.removeAt(els.size - 1)
      fun isEmpty() = els.isEmpty()
      fun size() = els.size
      override fun toString() = "Stack[${els.joinToString()}]"
  }

  fun isBalanced(sequence: String): Boolean {
      val stack: Stack<Char> = Stack()
      val map = mapOf(
          '(' to ')', ')' to '(',
          '[' to ']', ']' to '[',
          '{' to '}', '}' to '{'
      )
      
      // Write your solution here
  }

  fun main() {
      // Expected output:
      println(isBalanced("{([])}")) // true
      println(isBalanced("{([")) // false
  }
\end{verbatim}

\textbf{sequence} is valid if:
\begin{itemize}
  \item Open bracket must be closed by the same bracket type.
  \item Open bracket must be closed in the correct order.
\end{itemize}

\begin{verbatim}
  // Example 1
  Input: sequence = "()"
  Output: true
  
  // Example 2
  Input: sequence = "()[]{}"
  Output: true
  
  // Example 3
  Input: sequence = "{]"
  Output: false

  // Example 4
  Input: sequence = "{[}]"
  Output: false
\end{verbatim}

\subsection*{Problem 16 (2\%):}
Create a function which simulates the \textbf{Rock, Paper, Scissors} game. The function takes the input of both players (rock, paper or scissors), first parameter from the first player, second from the second player. The function returns the result as such:

\begin{itemize}
  \item First player wins
  \item Second player wins
  \item Draw
\end{itemize}

\begin{verbatim}
  // Examples
  rockPaperScissor("paper", "rock") => "First player wins"
  rockPaperScissor("rock", "paper") => "Second player wins"
  rockPaperScissor("paper", "paper") => "Draw"
\end{verbatim}

\subsection*{Problem 17 (1\%):}
You have been given a function called \textbf{isHappy} which accepts an \textbf{Int} parameter called \textbf{num}. Return \textbf{true} if \textbf{num} is a happy number \& \textbf{false} if not.\\

A happy number is a number defined by the following:
\begin{itemize}
  \item Starting with any positive integer, replace the number 
  \item Replace \textbf{num} by the sum of the squares of its digits.
  \item Repeat the process until \textbf{num} equals 1.
  \item If \textbf{num} equals 1, then it is happy.
\end{itemize}

\begin{verbatim}
  fun isHappy(n: Int): Boolean {
      // Write your solution here
  }

  fun main() {
      // Expected output:
      println(isHappy(19)) // true
      println(isHappy(2)) // false
  }
\end{verbatim}

\begin{verbatim}
  // Example 1
  Input: num = 19
  Output: true
  Breakdown:
  1 to the power of 2 + 9 to the power of 2 = 82
  8 to the power of 2 + 2 to the power of 2 = 68
  6 to the power of 2 + 8 to the power of 2 = 100
  1 to the power of 2 + 0 to the power of 2 + 0 to the power of 2 = 1
  
  // Example 2
  Input: num = 2
  Output: false
\end{verbatim}

\subsection*{Problem 18 (1\%):}
You have been given a function called \textbf{largestOddNumber} which accepts an \textbf{Int} parameter called \textbf{num}. Return the largest-valued odd integer or \textbf{-1} if no odd integer exists.

\begin{verbatim}
  fun largestOddNumber(num: Int): Int {
      // Write your solution here   
  }

  fun main() {
      // Expected output:
      println(largestOddNumber(19)) // 9
      println(largestOddNumber(35427)) // 35427
      println(largestOddNumber(1245)) // -1
  }
\end{verbatim}

\subsection*{Problem 19 (1\%):}
You have been given a function called \textbf{thirdLargest} which accepts an \textbf{String} parameter called \textbf{str}. Return the third largest numerical digit that appears in \textbf{str} or \textbf{-1} if it does not exist.

Given an alphanumeric string s, 
\begin{verbatim}
  fun thirdLargest(str: String): Int {
      // Write your solution here
  }

  fun main() {
      // Expected output:
      // println(thirdLargest("dfa12321afd")) // 3
      // println(thirdLargest("abc1111")) // -1
  }
\end{verbatim}

\subsection*{Problem 20 (1\%):}
You have been given a function called \textbf{uniqueSum} which accepts an \textbf{Array<Int>} parameter called \textbf{nums}. Return the sum of all the unique elements of \textbf{nums}.

\begin{verbatim}
  fun uniqueSum(nums: Array<Int>): Int {
      // Write your solution here 
  }

  fun main() {
      // Expected output:
      // println(uniqueSum([1, 2, 3, 2])) // 4
      // println(uniqueSum([1, 1, 1, 1, 1])) // 0
  }
\end{verbatim}

\end{document}