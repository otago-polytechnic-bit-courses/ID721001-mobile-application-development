% Author: Grayson Orr
% Course: ID721001: Mobile Application Development

\documentclass{article}
\author{}

\usepackage{graphicx}
\usepackage{wrapfig}
\usepackage{enumerate}
\usepackage{hyperref}
\usepackage[margin = 2.25cm]{geometry}
\usepackage[table]{xcolor}
\usepackage{fancyhdr}
\hypersetup{
  colorlinks = true,
  urlcolor = blue
}
\setlength\parindent{0pt}
\pagestyle{fancy}
\fancyhf{}
\rhead{College of Engineering, Construction and Living Sciences\\Bachelor of Information Technology}
\lfoot{Presentation Marks Breakdown\\Version 2, Semester One, 2022}
\rfoot{\thepage}
 
\begin{document}

\begin{figure}
	\centering
	\includegraphics[width=50mm]{../../resources/img/logo.png}
\end{figure}

\title{College of Engineering, Construction and Living Sciences\\Bachelor of Information Technology\\ID721001: Mobile Application Development\\Level 7, Credits 15\\\textbf{Presentation Marks Breakdown}}
\date{}
\maketitle

\subsection*{Documentation - Learning Outcomes 2, 3 (50\%)}
\begin{itemize}
	\item Documentation must contain the following sections:
	      \begin{itemize}
		      \item Overview - a brief description of what the topic is.
		      \item Dependencies - it may include the name, version number, etc. If it is not required, please indicate it appropriately.
		      \item Code example - a description of each code snippet in relation to the topic. It means you \textbf{only} have to describe the essential files.
		      \item References - the information in your documentation is referenced using \textbf{APA 7th edition}.
		            \begin{itemize}
			            \item \textbf{Resource:} \href{https://studentservices.op.ac.nz/learning-support/citingandreferencing}{https://studentservices.op.ac.nz/learning-support/citingandreferencing}
		            \end{itemize}
	      \end{itemize}
	\item Use of \textbf{Markdown}, i.e., bold text, code blocks, etc.
	\item Correct spelling \& grammar.
\end{itemize}

\subsection*{Presentation 2, 3 (50\%)}
\begin{itemize}
	\item Present your documentation, i.e., \textbf{README.md} via a video recording. In addition, you \textbf{must}:
	      \begin{itemize}
		      \item Upload your presentation to your \textbf{OP student OneDrive}.
		      \item Provide a link to your presentation in your documentation.
	      \end{itemize}
	\item Answer the following:
	      \begin{itemize}
		      \item Describe how would you implement it into your travelling \textbf{Project}.
	      \end{itemize}
\end{itemize}
\end{document}
