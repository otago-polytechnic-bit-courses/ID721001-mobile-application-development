% Author: Grayson Orr
% Course: ID721001: Mobile Application Development

\documentclass{article}
\author{}

\usepackage{graphicx}
\usepackage{wrapfig}
\usepackage{enumerate}
\usepackage{hyperref}
\usepackage[margin = 2.25cm]{geometry}
\usepackage[table]{xcolor}
\usepackage[super]{nth}
\hypersetup{
  colorlinks = true,
  urlcolor = blue
} 
\setlength\parindent{0pt} 

\begin{document}

\begin{figure}
	\includegraphics[width=50mm]{../../resources/img/logo.png}
\end{figure}

\title{Course Directive\\ID721001: Mobile Application Development\\Semester One, 2022}
\date{}
\maketitle

\section*{Course Information}
\begin{tabular}{ll}
	Credits:      & 15 Credits                                              \\
	Prerequisite: & IN/D607001: Introductory Application Development Concepts \\
	Timetable:    & Tuesday 8 AM D105b \& Thursday 8 AM D105b
\end{tabular}

\section*{Teaching Staff}
\begin{tabular}{ll}
	Name:            & Grayson Orr                             \\
	Position:        & Kaiako \& Second/Third-Year Coordinator \\
	Office Location: & D318                                    \\
	Email Address    & grayson.orr@op.ac.nz                    \\
\end{tabular}

\section*{Course Dates}
\begin{tabular}{ll}
	Term 1:             & Monday 21 February - Thursday 14 April \\
	Mid Semester Break: & Monday 18 April - Friday 29 April      \\
	Term 2:             & Monday 02 May - Thursday 23 June       \\
\end{tabular}

\section*{Public Holidays \& Anniversary Days}
A list of public holidays \& anniversary days can be found here - \href{https://www.op.ac.nz/students/importantdates}{https://www.op.ac.nz/students/importantdates}

\section*{Aims}
To learn the specifics of mobile application design \& development. Learners will be able to develop \& publish \textbf{Android} mobile applications using \textbf{Kotlin}, \textbf{Android Studio} \& \textbf{Google Play Store}.

\section*{Learning Outcomes}
At the successful completion of this course, learners will be able to:
\begin{enumerate}
	\item Implement \& publish complete, non-trivial, industry-standard mobile applications following sound architectural \& code-quality standards.
	\item Identify relevant use cases for a mobile computing scenario \& incorporate them into an effective user experience design.
	\item Follow industry standard software engineering practice in the design of mobile applications.
\end{enumerate}

\section*{Assessments}
\renewcommand{\arraystretch}{1.5}
\begin{tabular}{|c|c|c|c|}
	\hline
	\textbf{Assessment} & \textbf{Weight} & \textbf{Due Date}    & \textbf{Learning Outcomes} \\ \hline
	Project             & 65\%            & 10-06-2022 (Friday)  & 1, 2, 3                    \\ \hline
	Practicals          & 15\%            & 08-03-2022 (Tuesday) & 1, 2, 3                    \\ \hline
	Presentation        & 20\%            & 21-06-2022 (Tuesday) & 2, 3                       \\ \hline
\end{tabular} 

\section*{Provisional Schedule}

\begin{itemize}
	\item Online class - cyan highlight 
	\item No class - orange highlight
	\item \textbf{Assessment Work} is optional attendance
	\item Course \& teaching surveys will be emailed to you in \textbf{Week 12}
\end{itemize}

\renewcommand{\arraystretch}{1.5}
\begin{tabular}{|c|c|c|c|}
	\hline
	\textbf{Week}           & \textbf{Date}     & \multicolumn{2}{c|}{\textbf{Session}}                                             \\ \hline
	\small 1/Tahi           & \small 21-02-2022 & \small Kotlin 1                                   & \small Kotlin 2               \\ \hline
	\small 2/Rua            & \small 28-02-2022 & \multicolumn{2}{c|}{\cellcolor{cyan} Practicals Assessment Work}                                   \\ \hline
	\small 3/Toru           & \small 07-03-2022 & \small Android Overview                           & \cellcolor{cyan} \small Fragment               \\ \hline
	\small 4/Whā            & \small 14-03-2022 & \small ViewModel                                  & \small LiveData               \\ \hline
	\small 5/Rima           & \small 21-03-2022 & \cellcolor{orange} \small DataBinding                                & \cellcolor{cyan} \small Retrofit               \\ \hline
	\small 6/Ono            & \small 28-03-2022 & \small RecyclerView                               & \cellcolor{orange} \small Firebase Auth          \\ \hline
	\small 7/Whitu          & \small 04-04-2022 & \small Room Database                              & \cellcolor{cyan} \small Espresso               \\ \hline
	\small 8/Waru           & \small 11-04-2022 & \small DataStore                                  & \cellcolor{orange} Google Maps \\ \hline
	\rowcolor{yellow} \multicolumn{4}{|c|}{\footnotesize Mid Term Break}                                                            \\ \hline
	\small 9/Iwa            & \small 02-05-2022 & \cellcolor{orange} \small KDoc \& Dokka                              & \small Google Play Store      \\ \hline
	\small 10/Tekau         & \small 09-05-2022 & \multicolumn{2}{c|}{\small Project Assessment Work}                                      \\ \hline
	\small 11/Tekau mā tahi & \small 16-05-2022 & \multicolumn{2}{c|}{\small Project Assessment Work}                                      \\ \hline
	\small 12/Tekau mā rua  & \small 23-05-2022 & \multicolumn{2}{c|}{\small Project Assessment Work}                                      \\ \hline
	\small 13/Tekau mā toru & \small 30-05-2022 & \multicolumn{2}{c|}{\small Project Assessment Work}                                      \\ \hline
	\small 14/Tekau mā whā  & \small 06-06-2022 & \multicolumn{2}{c|}{\small Project Assessment Work}                                      \\ \hline
	\small 15/Tekau mā rima & \small 13-06-2022 & \multicolumn{2}{c|}{\small Presentation Assessment Work}                                 \\ \hline
	\small 16/Tekau mā ono  & \small 20-06-2022 & \multicolumn{2}{c|}{\small Presentation Assessment Work}                                 \\ \hline
\end{tabular}

\section*{Resources}

\subsection*{Software}
This paper will be taught using \textbf{Android Studio}. An installer for \textbf{Android Studio} are available. See \href{https://developer.android.com/studio/}{https://developer.android.com/studio}. Refer any problems with downloads or installers to Rob Broadley in D205a.

\subsection*{Readings}
No textbook is required for this course. URLs to useful resources will be provided in the lecture notes.

\section*{Course Requirements \& Expectations}

\subsection*{Learning Hours}
This course requires \textbf{150 hours} of learning. This time includes \textbf{10 hours} of meeting time, \& \textbf{140 hours} of self-directed reading, preparation \& completion of assessments.

\subsection*{Learning \& Teaching Methods}
From \textbf{Week Three} onwards, the lectured course material will be pre-recorded \& available to you via \textbf{Microsoft Teams}. You are \textbf{required} to view the recording prior to attending the class. Class time will consist of discussions \& application development work.  

\subsection*{Criteria for Passing}
To pass this paper, you must achieve a cumulative pass mark of \textbf{50\%} over all assessments. There are no reassessments or resits.

\subsection*{Attendance}
\begin{itemize}
	\item Learners are expected to attend all classes, including lectures \& labs.
	\item If you cannot attend for a few days for any reason, contact the course.
\end{itemize}

\subsection*{Communication}
\textbf{Microsoft Outlook/Teams} are the official communication channels for this course. It is your responsibility to regularly check \textbf{Microsoft Outlook/Teams} \& \href{https://github.com/otago-polytechnic-bit-courses/ID721001-mobile-application-development}{GitHub} for important course material, including changes to class scheduling or assessment details. Not checking will not be accepted as an excuse.

\subsection*{Snow Days/Polytechnic Closure}
In the event \textbf{Otago Polytechnic | Te Kura Matatini ki Otago} is closed or has a delayed opening because of snow or bad weather, you should not attempt to attend class if it is unsafe to do so. It is possible that the teaching staff will not be able to attend either, so classes will not physically be meeting. However, this does not become a holiday. Rather, the course material will be made available on \href{https://github.com/otago-polytechnic-bit-courses/ID721001-mobile-application-development}{GitHub} for classes affected by the closure. You are responsible for any course material presented in this manner. Information about closure will be posted on the \textbf{Otago Polytechnic | Te Kura Matatini ki Otago Facebook} page \href{https://www.facebook.com/OtagoPoly}{https://www.facebook.com/OtagoPoly}.

\subsection*{Group Work \& Originality}
Learners in the \textbf{Bachelor of Information Technology} programme are expected to hand in original work. Learners are encouraged to discuss assessments with their fellow learners, however, all assessments are to be completed as individual works unless group work is explicitly required (i.e. if it doesn't say it is group work then it is not group work - even if a group consultation was involved). Failure to submit your original work will be treated as plagiarism.

\subsection*{Referencing}
Appropriate referencing is required for all work. Referencing standards will be specified by the teaching staff.

\subsection*{Plagiarism}
Plagiarism is submitting someone elses work as your own. Plagiarism offences are taken seriously \& an assessment that has been plagiarised may be awarded a zero mark. A definition of plagiarism is in the Student Handbook, available online or at the school office.

\subsection*{Submission Requirements}
All assessments are to be submitted by the time, date, \& method given when the assessment is issued. Failure to meet all requirements will result in a penalty of up to \textbf{10\%} per day (including weekends).

\subsection*{Extensions}
Extensions are only available for unusual circumstances. These must be applied for, \& approved, before the submission date.

\subsection*{Impairment}
In case of sickness contact the teaching staff or \textbf{Head of Information Technology (Michael Holtz)} as soon as possible, preferably before the assessment is due. The policy regarding the granting of a mark that considers impaired performance requires a medical certificate \& a medical practitioner's signature on a form. You may refer to the guide on impaired performance on the student handbook.

\subsection*{Appeals}
If you are concerned about any aspect of your assessment, approach the teaching staff in the first instance. We support an open-door policy \& aim to resolve issues promptly. Further support is available from the \textbf{Head of Information Technology (Michael Holtz)} \& \textbf{Second/Third-Year Coordinator (Grayson Orr)}. \textbf{Otago Polytechnic | Te Kura Matatini ki Otago} has a formal process for academic appeals if necessary.

\subsection*{Other Documents}
Regulatory documents relating to this course can be found on the \textbf{Otago Polytechnic | Te Kura Matatini ki Otago} website.

\end{document}
