% Author: Grayson Orr
% Course: ID721001: Mobile Application Development

\documentclass{article}
\author{}

\usepackage{graphicx}
\usepackage{wrapfig}
\usepackage{enumerate}
\usepackage{hyperref}
\usepackage[margin = 2.25cm]{geometry}
\usepackage[table]{xcolor}
\usepackage{fancyhdr}
\hypersetup{
  colorlinks = true,
  urlcolor = blue
}
\setlength\parindent{0pt}
\pagestyle{fancy}
\fancyhf{}
\rhead{College of Engineering, Construction and Living Sciences\\Bachelor of Information Technology}
\lfoot{Presentation: Advanced Android Topic\\Version 3, Semester Two, 2022}
\rfoot{\thepage}
 
\begin{document}

\begin{figure}
	\centering
	\includegraphics[width=50mm]{../../resources/img/logo.png}
\end{figure} 

\title{College of Engineering, Construction and Living Sciences\\Bachelor of Information Technology\\ID721001: Mobile Application Development\\Level 7, Credits 15\\\textbf{Presentation: Advanced Android Topic}}
\date{}
\maketitle

\section*{Assessment Overview}
In this \textbf{individual} assessment, you will research, prepare \& present a mobile-related topic. The information presented must be in a \textbf{README.md} file. Also, you need to provide a code example to accompany the \textbf{README.md} file. The main purpose of this assessment is to demonstrate your ability to identify and effectively articulate an intermediate/advanced topic in \textbf{Android}.

\section*{Learning Outcomes}
At the successful completion of this course, learners will be able to:
\begin{enumerate}
	\item Implement \& publish complete, non-trivial, industry-standard mobile applications following sound architectural \& code-quality standards.
	\item Identify relevant use cases for a mobile computing scenario \& incorporate them into an effective user experience design.
	\item Follow industry standard software engineering practice in the design of mobile applications.
\end{enumerate}

\section*{Assessments} 
\renewcommand{\arraystretch}{1.5}
\begin{tabular}{|c|c|c|c|}
	\hline
	\textbf{Assessment} & \textbf{Weight} & \textbf{Due Date}    & \textbf{Learning Outcomes} \\ \hline
	Project: Travelling Application             & 60\%            & 28-10-2022 (Friday at 4.59 PM)  & 1, 2, 3                    \\ \hline
	Practical: Problem-Solving          & 20\%            & 26-08-2022 (Friday at 4.59 PM) & 1, 2, 3                    \\ \hline
	Presentation: Advanced Android Topic        & 20\%            & 16-11-2022 (Wednesday at 4.59 PM) & 2, 3                       \\ \hline
\end{tabular}

\section*{Conditions of Assessment}
You will complete this assessment during your learner-managed time. However, there will be time during class to discuss the requirements \& your progress on this assessment. This assessment will need to be completed by \textbf{Wednesday, 16 November 2022} at \textbf{4.59 PM}.

\section*{Pass Criteria}
This assessment is criterion-referenced (CRA) with a cumulative pass mark of \textbf{50\%} over all assessments in \textbf{ID721001: Mobile Application Development}.

\section*{Authenticity}
All parts of your submitted assessment \textbf{must} be completely your work. If you use code snippets from \textbf{GitHub}, \textbf{StackOverflow} or other online resources, you \textbf{must} reference it appropriately using \textbf{APA 7th edition}. Provide your references in the \textbf{README.md} file in your repository. Failure to do this will result in a mark of \textbf{zero} for this assessment.

\section*{Policy on Submissions, Extensions, Resubmissions \& Resits}
The school's process concerning submissions, extensions, resubmissions \& resits complies with \textbf{Otago Polytechnic} policies. Learners can view policies on the \textbf{Otago Polytechnic} website located at \href{https://www.op.ac.nz/about-us/governance-and-management/policies}{https://www.op.ac.nz/about-us/governance-and-management/policies}.

\section*{Submission}
You \textbf{must} submit all presentation files via \textbf{GitHub Classroom}. Here is the URL to the repository you will use for your submission - \href{https://classroom.github.com/a/Pfexjhjb}{https://classroom.github.com/a/Pfexjhjb}. The latest presentation files in the \textbf{master} or \textbf{main} branch will be used to mark against the \textbf{Documentation} criterion. Late submissions will incur a \textbf{10\% penalty per day}, rolling over at \textbf{5:00 PM}.

\section*{Extensions}
Familiarise yourself with the assessment due date. If you need an extension, contact the course lecturer before the due date. If you require more than a week's extension, a medical certificate or support letter from your manager may be needed.

\section*{Resubmissions}
Learners may be requested to resubmit an assessment following a rework of part/s of the original assessment. Resubmissions are to be completed within a negotiable short time frame \& usually \textbf{must} be completed within the timing of the course to which the assessment relates. Resubmissions will be available to learners who have made a genuine attempt at the first assessment opportunity \& achieved a \textbf{D grade (40-49\%)}. The maximum grade awarded for resubmission will be \textbf{C-}.

\section*{Resits}
Resits \& reassessments \textbf{are not} applicable in \textbf{ID721001: Mobile Application Development}.

\section*{Instructions}

List of topics:

\begin{itemize}
	\item Animations
	\item Biometric authentication
	\item CameraX
	\item Compose
	\item Dagger
	\item Environment sensors
	\item Hilt
	\item Location
	\item Media player
	\item Motion sensors
	\item Notifications
	\item Position sensors
	\item View binding
	\item View pager
	\item Work manager
\end{itemize}

\subsection*{Documentation - Learning Outcomes 2, 3 (50\%)}
\begin{itemize}
	\item Documentation must contain the following sections:
	      \begin{itemize}
		      \item Overview - a brief description of what the topic is.
		      \item Dependencies - it may include the name, version number, etc. If it is not required, please indicate it appropriately.
		      \item Code example - a description of each code snippet in relation to the topic. It means you \textbf{only} have to describe the essential files.
		      \item References - the information in your documentation is referenced using \textbf{APA 7th edition}.
		            \begin{itemize}
			            \item \textbf{Resource:} \href{https://studentservices.op.ac.nz/learning-support/citingandreferencing}{https://studentservices.op.ac.nz/learning-support/citingandreferencing}
		            \end{itemize}
	      \end{itemize}
	\item Use of \textbf{Markdown}, i.e., bold text, code blocks, etc.
	\item Correct spelling \& grammar.
\end{itemize}

\subsection*{Presentation 2, 3 (50\%)}
\begin{itemize}
	\item Present your documentation, i.e., \textbf{README.md} via a video recording. In addition, you \textbf{must}:
	      \begin{itemize}
		      \item Upload your presentation to your \textbf{OP student OneDrive}.
		      \item Provide a link to your presentation in your documentation.
	      \end{itemize}
	\item Answer the following:
	      \begin{itemize}
		      \item Describe how would you implement it into your travelling \textbf{Project}.
	      \end{itemize}
\end{itemize}

\subsection*{Additional Information}
\begin{itemize}
	\item Your presentation must not exceed \textbf{15 minutes} in length.
\end{itemize}
\end{document}
