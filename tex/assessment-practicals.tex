% Author: Grayson Orr

\documentclass{article}
\author{}

\usepackage{graphicx}
\usepackage{wrapfig}
\usepackage{enumerate}
\usepackage{hyperref}
\usepackage{float}
\usepackage[margin = 2.25cm]{geometry}
\usepackage[table]{xcolor}
\usepackage{fancyhdr}
\usepackage{makecell}
\usepackage{adjustbox}
\usepackage{pdflscape}
\renewcommand\cellgape{\Gape[3.5pt]}

\hypersetup{
  colorlinks = true,
  urlcolor = blue
}
\setlength\parindent{0pt}
\pagestyle{fancy}
\fancyhf{}
\rhead{College of Engineering, Construction and Living Sciences\\Bachelor of Information Technology}
\lfoot{Practicals \\Version 2, 2020}
\rfoot{\thepage}

\begin{document}

\begin{figure}
	\centering
	\includegraphics[width=50mm]{../resources/img/logo.png}
\end{figure}

\title{College of Engineering, Construction and Living Sciences\\Bachelor of Information Technology\\IN721: Design and Development of Applications for Mobile Devices\\Level 7, Credits 15\\\textbf{Practicals}}
\date{}
\maketitle

\section*{Assessment Table}
\renewcommand{\arraystretch}{1.5}
\begin{tabular}{|l|l|l|l|l|}
	\hline		
	\vtop{\hbox{\strut \textbf{Assessment}}\hbox{\strut \textbf{Activity}}} & \textbf{Weighting} & \vtop{\hbox{\strut \textbf{Learning}}\hbox{\strut \textbf{Outcomes}}} & \vtop{\hbox{\strut \textbf{Assessment}}\hbox{\strut \textbf{Grading Scheme}}} & \vtop{\hbox{\strut \textbf{Completion}}\hbox{\strut \textbf{Requirements}}} \\
							
	\hline
								
	\small Practicals                                                       & \small 10\%        & \small 1, 3, 4                                                        & \small CRA                                                                    & \small Cumulative                                                           \\ \hline
	\small Kotlin Travelling App                                            & \small 35\%        & \small 1, 3, 4                                                        & \small CRA                                                                    & \small Cumulative                                                           \\ \hline
	\small React Native Hacker News App                                     & \small 25\%        & \small 1, 3, 4                                                        & \small CRA                                                                    & \small Cumulative                                                           \\ \hline   
	\small Kotlin Exam                                    & \small 15\%        & \small 2, 3, 4                                                        & \small CRA                                                                    & \small Cumulative                                                           \\ \hline   
	\small React Native Exam                                     & \small 15\%        & \small 2, 3, 4                                                        & \small CRA                                                                    & \small Cumulative                                                           \\ \hline   
\end{tabular}

\section*{Conditions of Assessment}
This assessment will need to be completed by Wednesday, 11 November 2020 at 5pm.

\section*{Pass Criteria}
This assessment is criterion-referenced with a cumulative pass mark of 50\%.

\section*{Submission Details}
You must submit your program files via \textbf{GitHub Classroom}. Here is the link to the repository you will be using for your submission – \href{https://classroom.github.com/a/XfN8sJxf}{https://classroom.github.com/a/XfN8sJxf}.

\section*{Authenticity}
All parts of your submitted assessment must be completely your work and any references must be cited appropriately.

\section*{Policy on Submissions, Extensions, Resubmissions \& Resits}
The school's process concerning \textbf{Submissions, Extensions, Resubmissions and Resits} complies with Otago Polytechnic policies. Students can view policies on the Otago Polytechnic website located at \href{https://www.op.ac.nz/about-us/governance-and-management/policies}{https://www.op.ac.nz/about-us/governance-and-management/policies}.

\section*{Extensions}
Please familiarise yourself with the assessment due dates. If you need an extension, please contact your lecturer before the due date. If you require more than a week's extension, a medical certificate or support letter from your manager may be needed.

\section*{Resubmissions}
Students may be requested to resubmit an assessment following a rework of part/s of the original assessment. Resubmissions are completed within a short time frame (usually no more than 5 working days) and usually must be completed within the timing of the course to which the assessment relates. Resubmissions will be available to students who have made a genuine attempt at the first assessment opportunity. The maximum grade awarded for resubmission will be C-.

\section*{Learning Outcomes}
At the successful completion of this course, students will be able to: 
\begin{enumerate}
	\item Implement complete, non-trivial, industry-standard mobile applications following sound architectural and code-quality standards.
	\item Explain relevant principles of human perception and cognition and their importance to software design.
	\item Identify relevant use cases for a mobile computing scenario and incorporate them into an effective user experience design.
	\item Follow industry standard software engineering practice in the design of mobile applications.
\end{enumerate}

\newpage

\section*{Assessment Overview - Learning Outcomes 1, 3, 4}
In this assessment, you will complete a series of programming tasks covering the lecture \& resource material. \\

\renewcommand{\arraystretch}{1.5}
\begin{tabular}{|c|c|c|c|}
	\hline
	\textbf{Topic}                                                                                    & \textbf{Weighting}  & \textbf{Due Date} \\ \hline 
	\small Kotlin 1: Introduction to Android OS, Kotlin, Android Studio, Activity Lifecycle \& Intent & \small 1\%          & \small 02-09-2020 at 5pm \\ \hline
	\small Kotlin 2: Material Design, AsyncTask \& RecyclerView                                       & \small 1\%          & \small 02-09-2020 at 5pm \\ \hline
	\small Kotlin 3: Parcelable, CardView, SearchView \& SharedPreferences                            & \small 1\%          & \small 02-09-2020 at 5pm \\ \hline
	\small Kotlin 4: ProgressDialog, WebView, Fragment \& DialogFragment                              & \small 1\%          & \small 02-09-2020 at 5pm \\ \hline
	\small Kotlin 5: SQLite \& Location                                                               & \small 1\%          & \small 02-09-2020 at 5pm \\ \hline
	\small React Native Concepts                                  & \small 5\%          & \small 11-11-2020 at 5pm \\ \hline
\end{tabular}
\end{document}