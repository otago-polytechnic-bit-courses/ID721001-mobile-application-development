% Author: Grayson Orr

\documentclass{article}
\author{}

\usepackage{graphicx}
\usepackage{wrapfig}
\usepackage{enumerate}
\usepackage{hyperref}
\usepackage[margin = 2.25cm]{geometry}
\usepackage[table]{xcolor}
\usepackage{fancyhdr}
\hypersetup{
  colorlinks = true,
  urlcolor = blue
}
\setlength\parindent{0pt}
\pagestyle{fancy}
\fancyhf{}
\rhead{College of Engineering, Construction and Living Sciences\\Bachelor of Information Technology}
\lfoot{React Native Hacker News App\\Version 1, 2020}
\rfoot{\thepage}

\begin{document}

\begin{figure}
    \centering
    \includegraphics[width=50mm]{./img/logo.png}
\end{figure}

\title{College of Engineering, Construction and Living Sciences\\Bachelor of Information Technology\\IN721: Design and Development of Applications for Mobile Devices\\Level 7, Credits 15\\\textbf{React Native Hacker News App}}
\date{}
\maketitle

\section*{Assessment Overview}
For this assessment, you will develop \& publish a Hacker News application using React Native in Visual Studio Code \& Google Play Store. We won't be covering the basic features of React Native formally in class; you will be learning these features \textbf{independently}. The main purpose of this assessment is not just to build a simple application, rather demonstrate your ability to effectively learn a new technology which differs, both programmatically \& syntactically from Kotlin. In addition, marks will be allocated for application robustness, code elegance, documentation \& git usage. \\

The Hacker News application will help you keep up to date with the latest \& greatest news in computer science \& entrepreneurship. A user of your Hacker News application will be able to view \& read the 100 top stories, best stories \& job stories.

\section*{Assessment Table}
\renewcommand{\arraystretch}{1.5}
\begin{tabular}{|l|l|l|l|l|} 
    \hline      
    \vtop{\hbox{\strut \textbf{Assessment}}\hbox{\strut \textbf{Activity}}} & \textbf{Weighting} & \vtop{\hbox{\strut \textbf{Learning}}\hbox{\strut \textbf{Outcomes}}} & \vtop{\hbox{\strut \textbf{Assessment}}\hbox{\strut \textbf{Grading Scheme}}} & \vtop{\hbox{\strut \textbf{Completion}}\hbox{\strut \textbf{Requirements}}} \\
                            
    \hline
                                
    \small Practicals                                                       & \small 10\%        & \small 1, 3, 4                                                        & \small CRA                                                                    & \small Cumulative                                                           \\ \hline
    \small Kotlin Travelling App                                            & \small 35\%        & \small 1, 3, 4                                                        & \small CRA                                                                    & \small Cumulative                                                           \\ \hline
    \small React Native Hacker News App                                     & \small 25\%        & \small 1, 3, 4                                                        & \small CRA                                                                    & \small Cumulative                                                           \\ \hline   
    \small Kotlin Exam                                    & \small 15\%        & \small 2, 3, 4                                                        & \small CRA                                                                    & \small Cumulative                                                           \\ \hline   
    \small React Native Exam                                     & \small 15\%        & \small 2, 3, 4                                                        & \small CRA                                                                    & \small Cumulative                                                           \\ \hline   
\end{tabular}

\section*{Conditions of Assessment}
You will complete this assessment outside timetabled class time, however, there will be availability during the teaching sessions to discuss the requirements and progress of this assessment. Your application code must be written in \textbf{JavaScript}. Submitted \textbf{TypeScript} code will not be accepted. This assessment will need to be completed by Wednesday, 18 November 2020 at 5pm. 

\section*{Pass Criteria}
This assessment is criterion-referenced with a cumulative pass mark of 50\%.

\section*{Submission Details}
You must submit your program files via \textbf{GitHub Classroom}. Here is the link to the repository you will use for your submission – \href{https://classroom.github.com/a/5Ntput82}{https://classroom.github.com/a/5Ntput82}.

\section*{Group Contribution}
All git commit messages must identify which member(s) participated in the associated work session. Proportional contribution will be determined by inspection of the commit logs. If the commit logs show evidence of significantly uneven contribution proportion, the lecturer may choose to adjust the mark of the lesser contributor downward by an amount derived from the individual contributions.

\section*{Authenticity}
All parts of your submitted assessment must be completely your work and any references must be cited appropriately including, externally-sourced graphic elements. All media must be royalty free (or legally purchased) for educational use. Failure to do this will result in a mark of zero.

\section*{Policy on Submissions, Extensions, Resubmissions \& Resits}
The school's process concerning \textbf{Submissions, Extensions, Resubmissions and Resits} complies with Otago Polytechnic policies. Students can view policies on the Otago Polytechnic website located at \href{https://www.op.ac.nz/about-us/governance-and-management/policies}{https://www.op.ac.nz/about-us/governance-and-management/policies}.

\section*{Extensions}
Please familiarise yourself with the assessment due date. If you need an extension, please contact your lecturer before the due date. If you require more than a week's extension, a medical certificate or support letter from your manager may be needed.

\section*{Resubmissions}
Students may be requested to resubmit an assessment following a rework of part/s of the original assessment. Resubmissions are completed within a short time frame (usually no more than 5 working days) and usually must be completed within the timing of the course to which the assessment relates. Resubmissions will be available to students who have made a genuine attempt at the first assessment opportunity. The maximum grade awarded for resubmission will be C-.

\section*{Learning Outcomes}
At the successful completion of this course, students will be able to:
\begin{enumerate}
    \item Implement complete, non-trivial, industry-standard mobile applications following sound architectural and code-quality standards.
    \item Explain relevant principles of human perception and cognition and their importance to software design.
    \item Identify relevant use cases for a mobile computing scenario and incorporate them into an effective user experience design.
    \item Follow industry standard software engineering practice in the design of mobile applications.
\end{enumerate}

\newpage

\section*{Instructions} 

\subsection*{Functionality \& Robustness - Learning Outcomes 1, 3, 4}
\begin{itemize}
    \item Application must open without file structure modification in Visual Studio Code.
    \item Application must run without code modification on multiple mobile devices.
    \item Asynchronously fetch the first 100 top stories, best stories \& job stories from Hacker News using Axios \& the Hacker News API.
    \begin{itemize}
        \item \textbf{Resources:}
        \begin{itemize}
            \item \footnotesize\href{https://www.npmjs.com/package/axios}{https://www.npmjs.com/package/axios}
            \item \footnotesize\href{https://github.com/HackerNews/API}{https://github.com/HackerNews/API}
        \end{itemize}
    \end{itemize}
    \item For each story, display its title \& score from the response contents as an item in a flat list widget.
    \begin{itemize}
        \item \textbf{Resource:} \footnotesize\href{https://reactnative.dev/docs/flatlist.html}{https://reactnative.dev/docs/flatlist.html}
    \end{itemize}
    \item When a story item is clicked/pressed, display its URL from the response contents in a web view widget or redirect the user to the URL in the mobile device's default browser, i.e, Chrome.
    \begin{itemize}
        \item \textbf{Resources:} 
        \begin{itemize} 
            \item \footnotesize\href{https://www.npmjs.com/package/react-native-webview}{https://www.npmjs.com/package/react-native-webview}
            \item \footnotesize\href{https://reactnative.dev/docs/linking}{https://reactnative.dev/docs/linking}
        \end{itemize}
    \end{itemize}
    \item Bottom navigation widget which navigates the user to the appropriate collection of stories. You should have at least three menu icons, i.e., one for each story type. 
    \begin{itemize}
        \item \textbf{Resource:} \footnotesize\href{https://www.npmjs.com/package/react-native-material-bottom-navigation}{https://www.npmjs.com/package/react-native-material-bottom-navigation}
    \end{itemize}
    \item Splash screen with an image view widget \& transition animation.
    \begin{itemize}
        \item \textbf{Resources:}
        \begin{itemize}
            \item \footnotesize\href{https://reactnative.dev/docs/image}{https://reactnative.dev/docs/image}
            \item \footnotesize\href{https://reactnative.dev/docs/animations}{https://reactnative.dev/docs/animations}
        \end{itemize}
    \end{itemize}
    \item Adaptive launcher icon which displays a variety of shapes across different mobile devices.
    \begin{itemize}
        \item \textbf{Resource:} \footnotesize\href{https://romannurik.github.io/AndroidAssetStudio/icons-launcher.html}{https://romannurik.github.io/AndroidAssetStudio/icons-launcher.html} 
    \end{itemize}
    \item Visually attractive user-interface with a coherent graphical theme \& style.
    \item Application is published to Google Play Store.
    \begin{itemize}
        \item When you create the application, please name the package appropriately, for example, \\ \textbf{op.johndoe.hacker}. \textbf{Note:} replace \textbf{johndoe} with your Otago Polytechnic Ltd username.
        \item \textbf{Resources:}
        \begin{itemize}
            \item \footnotesize\href{https://docs.expo.io/workflow/publishing}{https://docs.expo.io/workflow/publishing}
            \item \footnotesize\href{https://docs.expo.io/distribution/uploading-apps}{https://docs.expo.io/distribution/uploading-apps}
        \end{itemize}
    \end{itemize}
    \item Ability to download the application from Google Play Store on to multiple mobile devices.
\end{itemize}

\subsection*{Documentation \& Git Usage - Learning Outcomes 3, 4}
\begin{itemize}
    \item Provide the following in the repository README file:
    \begin{itemize}
        \item How do you set up the environment for development, i.e., after the repository is cloned, what do I need to run the application?
        \item Privacy policy which discloses user information collected by the application.
        \item Sketched wireframes of the application. This can be hand-written or digitalised. \textbf{Note:} You must design you application before you start coding.
        \item Step-by-step user guide detailing each screen. The user guide must contain a screenshot of each screen in the application.
        \item Commented code is documented using JSDoc \& generated to \textbf{Markdown}.
        \begin{itemize}
            \item \textbf{Resources:} \footnotesize\href{https://jsdoc.app}{https://jsdoc.app}
        \end{itemize} 
    \end{itemize}
    \item At least 10 feature branches excluding the \textbf{main} branch.
    \begin{itemize}
        \item Your branches must be prefix with \textbf{feature}, for example, \textbf{feature-$<$name of functional requirement$>$}.
        \item For each branch, merge your own pull request to the \textbf{main} branch. 
    \end{itemize}
    \item Commit messages must reflect the context of each functional requirement change.
    \begin{itemize}
		\item \textbf{Resource:} \footnotesize\href{https://www.freecodecamp.org/news/writing-good-commit-messages-a-practical-guide}{https://www.freecodecamp.org/news/writing-good-commit-messages-a-practical-guide}
	\end{itemize}
\end{itemize}

\end{document}