% Author: Grayson Orr
% Course: IN721: Mobile Application Development

\documentclass{article}
\author{}

\usepackage{graphicx}
\usepackage{wrapfig}
\usepackage{enumerate}
\usepackage{hyperref}
\usepackage[margin = 2.25cm]{geometry}
\usepackage[table]{xcolor}
\usepackage[super]{nth}
\hypersetup{
  colorlinks = true,
  urlcolor = blue
}
\setlength\parindent{0pt}

\begin{document}

\begin{figure}
	\includegraphics[width=50mm]{./img/logo.png}
\end{figure}

\title{Course Directive\\IN721: Mobile Application Development\\Semester One, 2021}
\date{}
\maketitle

\section*{Course Information}
\begin{tabular}{ll}
	Credits:                      & 15 Credits                                                                   \\
	Prerequisite:                 & IN610: Programming 3 or IN607: Introductory Application Development Concepts \\
	Timetable:  Tuesday 8 AM D202 & Wednesday 10 AM D313                                                         \\
\end{tabular}

\section*{Lecturer}
\begin{tabular}{ll}
	Name:     & Grayson Orr (Lecturer) \\
	Location: & D311                   \\
	Email:    & grayson.orr@op.ac.nz   \\
\end{tabular}

\section*{Course Dates}
\begin{tabular}{ll}
	Term 1:             & 22 February - 16 April (8 weeks) \\
	Mid Semester Break: & 19 April - 30 April (2 weeks)    \\
	Term 2:             & 3 May - 25 June (8 weeks)        \\                       
\end{tabular}

\section*{Aims}
To learn the specifics of mobile application design \& development. Learners will be able to develop mobile applications using the Kotlin programming language \& publish them to Google Play Store.

\section*{Learning Outcomes}
At the successful completion of this course, learners will be able to: 
\begin{enumerate}
	\item Implement \& publish complete, non-trivial, industry-standard mobile applications following sound architectural \& code-quality standards.
	\item Identify relevant use cases for a mobile computing scenario \& incorporate them into an effective user experience design.
	\item Follow industry standard software engineering practice in the design of mobile applications.
\end{enumerate} 

\section*{Resources}

\subsection*{Software}
This paper will be taught using \textbf{Android Studio}. An installer for \textbf{Android Studio} is available. See \href{https://developer.android.com/studio}{https://developer.android.com/studio}. Please refer any problems with downloads or installers to \textbf{Rob Broadley} in \textbf{D205a}.

\subsection*{Readings}
There is no textbook for the course.

\section*{Provisional Schedule}

\renewcommand{\arraystretch}{1.5}
\begin{tabular}{|c|c|c|c|}
	\hline
	\textbf{Week} & \textbf{Date}     & \multicolumn{2}{c|}{\textbf{Session}}          \\ \hline
	\small 1      & \small 22-02-2021 & \multicolumn{2}{c|}{\small Kotlin}             \\ \hline
	\small 2      & \small 01-03-2021 & \multicolumn{2}{c|}{\small Age Calculator App} \\ \hline
	\small 3      & \small 08-03-2021 & \multicolumn{2}{c|}{\small Calculator App}     \\ \hline
	\small 4      & \small 15-03-2021 & \multicolumn{2}{c|}{\small Quiz App}           \\ \hline
	\small 5      & \small 22-03-2021 & \multicolumn{2}{c|}{\small Location App}       \\ \hline
	\small 6      & \small 29-03-2021 & \multicolumn{2}{c|}{\small Weather App}        \\ \hline
	\small 7      & \small 05-04-2021 & \multicolumn{2}{c|}{\small E-Commerce App}     \\ \hline
	\small 8      & \small 12-04-2021 & \multicolumn{2}{c|}{\small E-Commerce App}     \\ \hline
	\rowcolor{yellow} \multicolumn{4}{|c|}{\small Mid Term Break}  \\ \hline
	\small 9      & \small 03-05-2021 & \multicolumn{2}{c|}{\small Project Work}       \\ \hline
	\small 10     & \small 10-05-2021 & \multicolumn{2}{c|}{\small Project Work}       \\ \hline
	\small 11     & \small 17-05-2021 & \multicolumn{2}{c|}{\small Project Work}       \\ \hline
	\small 12     & \small 24-05-2021 & \multicolumn{2}{c|}{\small Project Work}       \\ \hline
	\small 13     & \small 31-05-2021 & \multicolumn{2}{c|}{\small Project Work}       \\ \hline
	\small 14     & \small 07-06-2021 & \multicolumn{2}{c|}{\small Project Work}       \\ \hline
	\small 15     & \small 14-06-2021 & \multicolumn{2}{c|}{\small Project Work}       \\ \hline
	\small 16     & \small 21-06-2021 & \multicolumn{2}{c|}{\small Project Work}       \\ \hline
\end{tabular}
 
\section*{Assessments}
\renewcommand{\arraystretch}{1.5}	
\begin{tabular}{|c|c|c|c|}
	\hline
	\textbf{Assessment} & \textbf{Weight} & \textbf{Due Date} & \textbf{Learning Outcomes} \\ \hline
	Practical           & 20\%            & 28-05-2021        & 2, 3                       \\ \hline
	Travelling App      & 40\%            & 23-06-2021        & 1, 2, 3                       \\ \hline
	Wishlist App        & 40\%            & 23-06-2021        & 1, 2, 3                       \\ \hline
\end{tabular} 


\section*{Course Requirements \& Expectations}

\subsection*{Learning Hours}
This course requires 150 hours of learning. This time includes 64 hours of timetabled class time, \& 86 hours of self-directed reading, preparation \& completion of assessment work.

\subsection*{Criteria for Passing}
To pass this paper, you must achieve an overall average of 50\%. There must be a genuine attempt at all assessments. There are no resits.

\subsection*{Attendance}
\begin{itemize}
	\item Learners are expected to attend all classes, including lectures \& labs.
	\item If you cannot attend for a few days for any reason, please contact your lecturer (Grayson Orr).
\end{itemize}

\subsection*{Communication}
Microsoft Outlook \& Teams are the official communication channels for this course. It is your responsibility to regularly check Microsoft Outlook/Teams \& \href{https://github.com/otago-polytechnic-bit-courses/IN721-mobile-app-dev}{GitHub} for important course-related material, including changes to class scheduling or assessment details. Not checking will not be accepted as an excuse.

\subsection*{Snow Days/Polytechnic Closure}
In the event the Polytechnic is closed or has a delayed opening because of snow or bad weather, you should not attempt to attend class if it is unsafe to do so. It is possible that your lecturer (Grayson Orr) will not be able to attend either, so classes will not physically be meeting. However, this does not become a holiday. Rather, the material will be made available on \href{https://github.com/otago-polytechnic-bit-courses/IN721-mobile-app-dev}{GitHub} for classes affected by the closure. You are responsible for any material presented in this manner. Information about closure will be posted on the Otago Polytechnic Facebook page \href{https://www.facebook.com/OtagoPoly}{https://www.facebook.com/OtagoPoly}.

\subsection*{Group Work \& Originality}
Learners in the Bachelor of Information Technology degree are expected to hand in original work. Learners are encouraged to discuss assessments with their fellow learners, however, all assessments are to be completed as individual works unless group work is explicitly required (i.e. if it doesn’t say it is group work then it is not group work – even if a group consultation was involved). Failure to submit your original work will be treated as plagiarism.

\subsection*{Referencing}
Appropriate referencing is required for all work. Referencing standards will be specified by your lecturer (Grayson Orr).

\subsection*{Plagiarism}
Plagiarism is submitting someone elses work as your own. Plagiarism offences are taken seriously \& an assessment that has been plagiarised may be awarded a zero mark. A definition of plagiarism is in the Student Handbook, available online or at the school office.

\subsection*{Submission Requirements}
All assessments are to be submitted by the time, date, \& method given when the assessment is issued. Failure to meet all requirements may result in a penalty of up to 10\% per day (including weekends).

\subsection*{Extensions}
Extensions are only available for unusual circumstances. These must be applied for, \& approved, before the submission deadline.

\subsection*{Impairment}
In case of sickness contact your lecturer (Grayson Orr) or BIT Team Leader (Michael Holtz) as soon as possible, preferably before the assessment or exam is due. The policy regarding the granting of a mark that considers impaired performance requires a medical certificate \& a medical practitioner’s signature on a form. You may refer to the guide on impaired performance on the student handbook.

\subsection*{Appeals}
If you are concerned about any aspect of your assessment, please approach your lecturer (Grayson Orr) in the first instance. We support an open-door policy \& aim to resolve issues promptly. Further support is available from the BIT Team Leader (Michael Holtz) \& Head of College (Richard Nyhof). Otago Polytechnic has a formal process for academic appeals if necessary.

\subsection*{Other Documents}
Regulatory documents relating to this course can be found on the Otago Polytechnic website.

\end{document}