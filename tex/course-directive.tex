% Author: Grayson Orr
% Course: IN721: Mobile Application Development

\documentclass{article}
\author{}

\usepackage{graphicx}
\usepackage{wrapfig}
\usepackage{enumerate}
\usepackage{hyperref}
\usepackage[margin = 2.25cm]{geometry}
\usepackage[table]{xcolor}
\usepackage[super]{nth}
\hypersetup{
  colorlinks = true,
  urlcolor = blue
} 
\setlength\parindent{0pt}

\begin{document} 

\begin{figure}
	\includegraphics[width=50mm]{./img/logo.png}
\end{figure} 

\title{Course Directive\\IN721: Mobile Application Development\\Semester One, 2021}
\date{}
\maketitle

\section*{Course Information}
\begin{tabular}{ll}
	Credits:                      & 15 Credits                                                                   \\
	Prerequisite:                 & IN610: Programming 3 or IN607: Introductory Application Development Concepts \\
	Timetable:  & Tuesday 8 AM D105b \& Wednesday 10 AM D313                                                         \\
\end{tabular}

\section*{Lecturer}
\begin{tabular}{ll}
	Name:     & Grayson Orr (Lecturer) \\
	Location: & D311                   \\
	Email:    & grayson.orr@op.ac.nz   \\
\end{tabular}

\section*{Course Dates}
\begin{tabular}{ll}
	Term 1:             & 22 February - 16 April (8 weeks) \\
	Mid Semester Break: & 19 April - 30 April (2 weeks)    \\
	Term 2:             & 03 May - 25 June (8 weeks)        \\     
	Easter Tuesday:     & 06 April                             \\                 
\end{tabular}

\section*{Aims}
To learn the specifics of mobile application design \& development. Learners will be able to develop \& publish \textbf{Android} mobile applications using \textbf{Kotlin}, \textbf{Android Studio} \& \textbf{Google Play Store}.

\section*{Learning Outcomes}
At the successful completion of this course, learners will be able to: 
\begin{enumerate}
	\item Implement \& publish complete, non-trivial, industry-standard mobile applications following sound architectural \& code-quality standards.
	\item Identify relevant use cases for a mobile computing scenario \& incorporate them into an effective user experience design.
	\item Follow industry standard software engineering practice in the design of mobile applications.
\end{enumerate} 

\section*{Assessments}
\renewcommand{\arraystretch}{1.5}	
\begin{tabular}{|c|c|c|c|}
	\hline
	\textbf{Assessment} & \textbf{Weight} & \textbf{Due Date} & \textbf{Learning Outcomes} \\ \hline
	Practical           & 20\%            & Various           & 2, 3                       \\ \hline
	Project             & 80\%            & 23-06-2021        & 1, 2, 3                    \\ \hline
\end{tabular} 

\section*{Provisional Schedule}
\renewcommand{\arraystretch}{1.5}
\begin{tabular}{|c|c|c|c|}
	\hline
	\textbf{Week} & \textbf{Date}     & \multicolumn{2}{c|}{\textbf{Session}}          \\ \hline
	\small 1 & \small 22-02-2021 & Kotlin            						& Kotlin (Online)									\\ \hline
	\small 2 & \small 01-03-2021 & Android Overview             & Event Handling \\ \hline
	\small 3 & \small 08-03-2021 & Data Passing (Online)        & UI Testing (Self-Directed)     \\ \hline
	\small 4 & \small 15-03-2021 & Fragments                    & ViewModel      \\ \hline
	\small 5 & \small 22-03-2021 & LiveData (Online)            & Binding   \\ \hline
	\small 6 & \small 29-03-2021 & Transformations     & Google Play Store (Pre-Recorded)        \\ \hline
	\small 7 & \small 05-04-2021 & Room Database (Pre-Recorded) & Retrofit       \\ \hline
	\small 8 & \small 12-04-2021 & RecyclerView                 & Google Maps    \\ \hline
	\rowcolor{yellow} \multicolumn{4}{|c|}{\small Mid Term Break}  \\ \hline
	\small 9 & \small 03-05-2021 & \multicolumn{2}{c|}{\small Project Work}       \\ \hline
	\small 10     & \small 10-05-2021 & \multicolumn{2}{c|}{\small Project Work}        \\ \hline
	\small 11     & \small 17-05-2021 & \multicolumn{2}{c|}{\small Project Work}       \\ \hline
	\small 12     & \small 24-05-2021 & \multicolumn{2}{c|}{\small Project Work}       \\ \hline
	\small 13     & \small 31-05-2021 & \multicolumn{2}{c|}{\small Project Work}       \\ \hline
	\small 14     & \small 07-06-2021 & \multicolumn{2}{c|}{\small Project Work}       \\ \hline
	\small 15     & \small 14-06-2021 & \multicolumn{2}{c|}{\small Project Work}       \\ \hline
	\small 16     & \small 21-06-2021 & \multicolumn{2}{c|}{\small Project Work}       \\ \hline
\end{tabular}

\section*{Resources}

\subsection*{Software}
This paper will be taught using \textbf{Android Studio} \& \textbf{IntelliJ IDEA}. An installer for \textbf{Android Studio} \& \textbf{IntelliJ IDEA} are available. See \href{https://developer.android.com/studio/}{https://developer.android.com/studio} \& \href{https://www.jetbrains.com/idea/download/}{https://www.jetbrains.com/idea/download}. Refer any problems with downloads or installers to Rob Broadley in D205a.

\subsection*{Readings}
There is no textbook for the course.

\section*{Course Requirements \& Expectations}

\subsection*{Learning Hours}
This course requires \textbf{150 hours} of learning. This time includes \textbf{64 hours} of timetabled class time, \& \textbf{86 hours} of self-directed reading, preparation \& completion of assessments.

\subsection*{Criteria for Passing}
To pass this paper, you must achieve a cumulative pass mark of \textbf{50\%} over all assessments. There are no reassessments or resits.

\subsection*{Attendance}
\begin{itemize}
	\item Learners are expected to attend all classes, including lectures \& labs.
	\item If you cannot attend for a few days for any reason, contact the course.
\end{itemize}

\subsection*{Communication}
\textbf{Microsoft Outlook/Teams} are the official communication channels for this course. It is your responsibility to regularly check \textbf{Microsoft Outlook/Teams} \& \href{https://github.com/otago-polytechnic-bit-courses/IN721-mobile-application-development}{GitHub} for important course material, including changes to class scheduling or assessment details. Not checking will not be accepted as an excuse.

\subsection*{Snow Days/Polytechnic Closure}
In the event \textbf{Otago Polytechnic} is closed or has a delayed opening because of snow or bad weather, you should not attempt to attend class if it is unsafe to do so. It is possible that the course lecturer will not be able to attend either, so classes will not physically be meeting. However, this does not become a holiday. Rather, the course material will be made available on \href{https://github.com/otago-polytechnic-bit-courses/IN721-mobile-application-development}{GitHub} for classes affected by the closure. You are responsible for any course material presented in this manner. Information about closure will be posted on the \textbf{Otago Polytechnic Facebook} page \href{https://www.facebook.com/OtagoPoly}{https://www.facebook.com/OtagoPoly}.

\subsection*{Group Work \& Originality}
Learners in the \textbf{Bachelor of Information Technology} programme are expected to hand in original work. Learners are encouraged to discuss assessments with their fellow learners, however, all assessments are to be completed as individual works unless group work is explicitly required (i.e. if it doesn’t say it is group work then it is not group work – even if a group consultation was involved). Failure to submit your original work will be treated as plagiarism.

\subsection*{Referencing}
Appropriate referencing is required for all work. Referencing standards will be specified by the course lecturer.

\subsection*{Plagiarism}
Plagiarism is submitting someone elses work as your own. Plagiarism offences are taken seriously \& an assessment that has been plagiarised may be awarded a zero mark. A definition of plagiarism is in the Student Handbook, available online or at the school office.

\subsection*{Submission Requirements}
All assessments are to be submitted by the time, date, \& method given when the assessment is issued. Failure to meet all requirements will result in a penalty of up to \textbf{10\%} per day (including weekends).

\subsection*{Extensions}
Extensions are only available for unusual circumstances. These must be applied for, \& approved, before the submission date.

\subsection*{Impairment}
In case of sickness contact the course lecturer or \textbf{BIT Team Leader (Michael Holtz)} as soon as possible, preferably before the assessment is due. The policy regarding the granting of a mark that considers impaired performance requires a medical certificate \& a medical practitioner’s signature on a form. You may refer to the guide on impaired performance on the student handbook.

\subsection*{Appeals}
If you are concerned about any aspect of your assessment, approach the course lecturer in the first instance. We support an open-door policy \& aim to resolve issues promptly. Further support is available from the \textbf{BIT Team Leader (Michael Holtz)} \& \textbf{Head of College (Richard Nyhof)}. \textbf{Otago Polytechnic} has a formal process for academic appeals if necessary.

\subsection*{Other Documents}
Regulatory documents relating to this course can be found on the \textbf{Otago Polytechnic} website.

\end{document}