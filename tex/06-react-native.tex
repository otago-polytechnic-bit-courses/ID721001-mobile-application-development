% Author: Grayson Orr

\documentclass{article}
\author{}

\usepackage{graphicx}
\usepackage{wrapfig}
\usepackage{enumerate}
\usepackage{hyperref}
\usepackage[margin = 2.25cm]{geometry}
\usepackage[table]{xcolor}
\usepackage{fancyhdr}
\hypersetup{
  colorlinks = true,
  urlcolor = blue
}
\setlength\parindent{0pt}
\pagestyle{fancy}
\fancyhf{}
\rhead{College of Engineering, Construction and Living Sciences\\Bachelor of Information Technology}
\lfoot{Practical 06: React Native\\Version 1, 2020}
\rfoot{\thepage}

\begin{document}

\begin{figure}
    \centering
    \includegraphics[width=50mm]{img/logo.png}
\end{figure}

\title{College of Engineering, Construction and Living Sciences\\Bachelor of Information Technology\\IN721: Design and Development of Applications for Mobile Devices\\Level 7, Credits 15\\\textbf{Practical 06: React Native}}
\date{}
\maketitle

\textbf{Due Date:} 27/10/2020 at 5pm \\

In this practical, you will complete a series of tasks covering today's lecture. This practical is worth 5\% of the final mark for the IN721: Design and Development of Applications for Mobile Devices course. \\

Before you start, in your practicals repository, create a new branch called \textbf{06-submission}. 

\section*{Task 1 - 2\%} 
Create a new \textbf{React Native} application with \textbf{Expo} using the following command:
\begin{verbatim}
    expo init <APP_NAME>
\end{verbatim}

In \texttt{App.js}, use \texttt{Axios} to fetch data from the \href{https://api.nasa.gov/planetary/apod?api\_key=DEMO\_KEY}{Astronomy Picture of the Day} endpoint. \textbf{Note:} \texttt{DEMO\_KEY} is the API key which you will need to generate yourself. Display the \texttt{title}, \texttt{date}, \texttt{explanation} \& \texttt{url} in an appropriate view. Please ensure you display \texttt{url} in an \texttt{Image}.

\section*{Task 2 - 3\%} 
Create a new \textbf{React Native} application with \textbf{Expo}  or use the same application from \textbf{Task 1}. In \texttt{App.js}, use \texttt{Axios} to fetch data from the \href{https://api.github.com/users/GITHUB\_USERNAME}{GitHub Username} endpoint. Add a \texttt{TextInput} \& \texttt{Button}. When the user enters an username into the \texttt{TextInput} \& clicks the \texttt{Button}, it will make a request to the endpoint. Display the \texttt{name}, \texttt{bio}, \texttt{avatar\_url} in an appropriate view. Please ensure correct error checking, i.e., username is not found.

\subsection*{Resources} 
\begin{itemize}
  \item \href{https://reactnative.dev/}{React Native}
  \item \href{https://expo.io/}{Expo}
  \item \href{https://www.npmjs.com/package/axios}{Axios}
  \item \href{https://developer.github.com/v3/}{GitHub API}
  \item \href{https://api.nasa.gov/}{NASA Open API}
\end{itemize}
 
\end{document}