% Author: Grayson Orr
% Course: IN721: Design and Development of Applications for Mobile Devices

\documentclass{article}
\author{}

\usepackage{graphicx}
\usepackage{wrapfig}
\usepackage{enumerate}
\usepackage{listings}
\usepackage{hyperref}
\usepackage[margin = 2.25cm]{geometry}
\usepackage[table]{xcolor}
\usepackage{fancyhdr}
\usepackage{lscape}
\usepackage{float}
\usepackage{enumitem}
\hypersetup{
  colorlinks = true,
  urlcolor = blue
}
\setlength\parindent{0pt}
\pagestyle{fancy}
\fancyhf{}
\rhead{College of Engineering, Construction and Living Sciences\\Bachelor of Information Technology}
\lfoot{React Native Exam \\Version 1, 2020}
\rfoot{\thepage}

\begin{document}

\begin{figure}
	\centering
	\includegraphics[width=50mm]{./img/logo.png} 
\end{figure}

\title{College of Engineering, Construction and Living Sciences\\Bachelor of Information Technology\\IN721: Design and Development of Applications for Mobile Devices\\Level 7, Credits 15\\\textbf{React Native Exam}}
\date{}
\maketitle

\section*{Assessment Overview}
In this assessment, you will undertake an exam covering React Native. This assessment is worth 15\% of the final mark for the Design and Development of Applications for Mobile Devices course. 

\section*{Assessment Table}
\renewcommand{\arraystretch}{1.5}
\begin{tabular}{|l|l|l|l|l|}
	\hline		
	\vtop{\hbox{\strut \textbf{Assessment}}\hbox{\strut \textbf{Activity}}} & \textbf{Weighting} & \vtop{\hbox{\strut \textbf{Learning}}\hbox{\strut \textbf{Outcomes}}} & \vtop{\hbox{\strut \textbf{Assessment}}\hbox{\strut \textbf{Grading Scheme}}} & \vtop{\hbox{\strut \textbf{Completion}}\hbox{\strut \textbf{Requirements}}} \\
							
	\hline
								
	\small Practicals                                                       & \small 10\%        & \small 1, 3, 4                                                        & \small CRA                                                                    & \small Cumulative                                                           \\ \hline
	\small Kotlin Travelling App                                            & \small 35\%        & \small 1, 3, 4                                                        & \small CRA                                                                    & \small Cumulative                                                           \\ \hline
	\small React Native Hacker News App                                     & \small 25\%        & \small 1, 3, 4                                                        & \small CRA                                                                    & \small Cumulative                                                           \\ \hline   
	\small Kotlin Exam                                    & \small 15\%        & \small 2, 3, 4                                                        & \small CRA                                                                    & \small Cumulative                                                           \\ \hline   
	\small React Native Exam                                     & \small 15\%        & \small 2, 3, 4                                                        & \small CRA                                                                    & \small Cumulative                                                           \\ \hline   
\end{tabular}

\section*{Conditions of Assessment}
This is an open-book/open-internet assessment. All answers should be in your own writing. Please do not directly copy an answer from the internet. If you do, this will result in an exam mark of zero. There is no word limit / word count for this assessment, however, it is important to be succinct \& concise when answering. Please do not provide code as an answer. If you do, this will result in an question mark of zero. \\ 

There are 50 marks in total. This assessment will need to be completed by Monday, 9 November 2020 at 10 am. 

\section*{Pass Criteria}
This assessment is criterion-referenced with a cumulative pass mark of 50\%.

\section*{Submission Details}
You must submit your exam via GitHub Classroom. Here is the link to the repository you will use for your submission – \href{https://classroom.github.com/a/CgfXMhqk}{https://classroom.github.com/a/CgfXMhqk}. Your submission must be presented in a PDF format. 

\section*{Policy on Submissions, Extensions, Resubmissions \& Resits}
The school's process concerning \textbf{Submissions, Extensions, Resubmissions and Resits} complies with Otago Polytechnic policies. Students can view policies on the Otago Polytechnic website located at \href{https://www.op.ac.nz/about-us/governance-and-management/policies}{https://www.op.ac.nz/about-us/governance-and-management/policies}.

\section*{Extensions}
Extension is unavailable for this assessment. 

\section*{Resubmissions}
Resubmission is unavailable for this assessment.

\section*{Resits} 
Resit is unavailable for this assessment.

\section*{Learning Outcomes}
At the successful completion of this course, students will be able to:
\begin{enumerate}
	\item Implement complete, non-trivial, industry-standard mobile applications following sound architectural and code-quality standards.
	\item Explain relevant principles of human perception and cognition and their importance to software design.
	\item Identify relevant use cases for a mobile computing scenario and incorporate them into an effective user experience design.
	\item Follow industry standard software engineering practice in the design of mobile applications.
\end{enumerate}

\newpage

\section*{Learning Outcomes: 2, 3, 4}

\subsection*{Question 1 (3 marks)}
Describe what \textbf{React Hooks} is.

\subsection*{Question 2 (4 marks)}
Describe what the rules are for using \textbf{React Hooks}.

\subsection*{Question 3 (3 marks)}
What is a \textbf{component} \& what is it used for? Carefully describe the difference between a \textbf{native} component \& a \textbf{core} component.

\subsection*{Question 4 (4 marks):}
Describe \textbf{five} benefits of using \textbf{React Native} to build mobile applications?

\subsection*{Question 5 (3 marks)}
Briefly describe what \textbf{JSX} is.

\subsection*{Question 6 (2 marks):}
What is \textbf{state} in a \textbf{React} component?

\subsection*{Question 7 (3 marks):}
What is the purpose of \textbf{StyleSheet.create()}?

\subsection*{Question 8 (4 marks):}
What is the difference between \textbf{Shadow DOM} and \textbf{Virtual DOM}?

\subsection*{Question 9 (2 marks):}
What is \textbf{props} in a \textbf{React} component?

\subsection*{Question 10 (4 marks):}
Describe \textbf{four} limitations of \textbf{React Native}?

\subsection*{Question 11 (3 marks)}
Carefully describe the difference between \textbf{SQLite} \& \textbf{Async Storage}.

\subsection*{Question 12 (4 marks)}
How does \textbf{React Native} manage various screen sizes?

\subsection*{Question 13 (3 marks)}
Describe the difference between \textbf{controlled} \& \textbf{uncontrolled} components?

\subsection*{Question 14 (5 marks):}
Describe how you would \textbf{asynchronously} fetch data from an API, i.e., NASA API, \& display the response contents on the mobile screen.

\subsection*{Question 15 (3 marks):}
Describe what \textbf{presentational} or \textbf{dumb} components are.

\end{document}