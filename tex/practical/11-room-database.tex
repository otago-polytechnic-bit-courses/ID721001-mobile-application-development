% Author: Grayson Orr
% Course: IN721: Mobile Application Development

\documentclass{article}
\author{}

\usepackage{graphicx}
\usepackage{wrapfig}
\usepackage{enumerate}
\usepackage{hyperref}
\usepackage[margin = 2.25cm]{geometry}
\usepackage[table]{xcolor}
\usepackage{fancyhdr}
\hypersetup{
  colorlinks = true,
  urlcolor = blue
}
\setlength\parindent{0pt}
\pagestyle{fancy}
\fancyhf{}
\rhead{College of Engineering, Construction and Living Sciences\\Bachelor of Information Technology}
\lfoot{Practical 11: Room Database\\Version 1, Semester One, 2020}
\rfoot{\thepage}

\begin{document}

\begin{figure}
    \centering
    \includegraphics[width=50mm]{../img/logo.png}
\end{figure}

\title{College of Engineering, Construction and Living Sciences\\Bachelor of Information Technology\\IN721: Mobile Application Development\\Level 7, Credits 15\\\textbf{Practical 11: Room Database}}
\date{}
\maketitle

\section*{Learning Outcomes}
At the successful completion of this course, learners will be able to: 
\begin{enumerate}
	\item Implement \& publish complete, non-trivial, industry-standard mobile applications following sound architectural \& code-quality standards.
	\item Identify relevant use cases for a mobile computing scenario \& incorporate them into an effective user experience design.
	\item Follow industry standard software engineering practice in the design of mobile applications.
\end{enumerate} 


\section*{Assessment Table}
\renewcommand{\arraystretch}{1.5}	
\begin{tabular}{|c|c|c|c|}
	\hline
	\textbf{Assessment} & \textbf{Weight} & \textbf{Due Date} & \textbf{Learning Outcomes} \\ \hline
	Practical           & 20\%            & Various           & 2, 3                       \\ \hline
	Project             & 80\%            & 23-06-2021        & 1, 2, 3                    \\ \hline
\end{tabular} 

 
\end{document}