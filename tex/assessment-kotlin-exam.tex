% Author: Grayson Orr
% Course: IN721: Design and Development of Applications for Mobile Devices

\documentclass{article}
\author{}

\usepackage{graphicx}
\usepackage{wrapfig}
\usepackage{enumerate}
\usepackage{listings}
\usepackage{hyperref}
\usepackage[margin = 2.25cm]{geometry}
\usepackage[table]{xcolor}
\usepackage{fancyhdr}
\usepackage{lscape}
\usepackage{float}
\usepackage{enumitem}
\hypersetup{
  colorlinks = true,
  urlcolor = blue
}
\setlength\parindent{0pt}
\pagestyle{fancy}
\fancyhf{}
\rhead{College of Engineering, Construction and Living Sciences\\Bachelor of Information Technology}
\lfoot{Kotlin Exam \\Version 1, 2020}
\rfoot{\thepage}

\begin{document}

\begin{figure}
	\centering
	\includegraphics[width=50mm]{./img/logo.png} 
\end{figure}

\title{College of Engineering, Construction and Living Sciences\\Bachelor of Information Technology\\IN721: Design and Development of Applications for Mobile Devices\\Level 7, Credits 15\\\textbf{Kotlin Exam}}
\date{}
\maketitle

\section*{Assessment Overview}
In this assessment, you will undertake an exam covering the Kotlin lectures, practicals \& assignment. This assessment is worth 15\% of the final mark for the Design and Development of Applications for Mobile Devices course. 

\section*{Assessment Table}
\renewcommand{\arraystretch}{1.5}
\begin{tabular}{|l|l|l|l|l|}
	\hline		
	\vtop{\hbox{\strut \textbf{Assessment}}\hbox{\strut \textbf{Activity}}} & \textbf{Weighting} & \vtop{\hbox{\strut \textbf{Learning}}\hbox{\strut \textbf{Outcomes}}} & \vtop{\hbox{\strut \textbf{Assessment}}\hbox{\strut \textbf{Grading Scheme}}} & \vtop{\hbox{\strut \textbf{Completion}}\hbox{\strut \textbf{Requirements}}} \\
							
	\hline
								
	\small Practicals                                                       & \small 10\%        & \small 1, 3, 4                                                        & \small CRA                                                                    & \small Cumulative                                                           \\ \hline
	\small Kotlin Travelling App                                            & \small 35\%        & \small 1, 3, 4                                                        & \small CRA                                                                    & \small Cumulative                                                           \\ \hline
	\small React Native Hacker News App                                     & \small 25\%        & \small 1, 3, 4                                                        & \small CRA                                                                    & \small Cumulative                                                           \\ \hline   
	\small Kotlin Exam                                    & \small 15\%        & \small 2, 3, 4                                                        & \small CRA                                                                    & \small Cumulative                                                           \\ \hline   
	\small React Native Exam                                     & \small 15\%        & \small 2, 3, 4                                                        & \small CRA                                                                    & \small Cumulative                                                           \\ \hline   
\end{tabular}

\section*{Conditions of Assessment}
This is an open-book/open-internet assessment. All answers should be in your own writing. Please do not directly copy an answer from the internet. If you do, this will result in an exam mark of zero. There is no word limit / word count for this assessment, however, it is important to be succinct \& concise when answering. Please do not provide code as an answer. If you do, this will result in an question mark of zero. \\ 

There are 50 marks in total. This assessment will need to be completed by Monday, 21 September 2020 at 10 am.

\section*{Pass Criteria}
This assessment is criterion-referenced with a cumulative pass mark of 50\%.

\section*{Submission Details}
You must submit your exam via GitHub Classroom. Here is the link to the repository you will use for your submission – \href{https://classroom.github.com/a/g4nE9aQ-}{https://classroom.github.com/a/g4nE9aQ-}. Your submission must be presented in a PDF format. 

\section*{Policy on Submissions, Extensions, Resubmissions \& Resits}
The school's process concerning \textbf{Submissions, Extensions, Resubmissions and Resits} complies with Otago Polytechnic policies. Students can view policies on the Otago Polytechnic website located at \href{https://www.op.ac.nz/about-us/governance-and-management/policies}{https://www.op.ac.nz/about-us/governance-and-management/policies}.

\section*{Extensions}
Extension is unavailable for this assessment. 

\section*{Resubmissions}
Resubmission is unavailable for this assessment.

\section*{Resits} 
Resit is unavailable for this assessment.

\section*{Learning Outcomes}
At the successful completion of this course, students will be able to:
\begin{enumerate}
	\item Implement complete, non-trivial, industry-standard mobile applications following sound architectural and code-quality standards.
	\item Explain relevant principles of human perception and cognition and their importance to software design.
	\item Identify relevant use cases for a mobile computing scenario and incorporate them into an effective user experience design.
	\item Follow industry standard software engineering practice in the design of mobile applications.
\end{enumerate}

\newpage

\section*{Learning Outcomes: 2, 3, 4}

\subsection*{Question 1 (5 marks):}
What is an \textbf{Intent} \& what is it used for? Carefully describe the difference between an \textbf{explicit} intent \& an \textbf{implict} intent.

\subsection*{Question 2 (3 marks):}
When is the \textbf{onStop()} lifecycle method invoked?

\subsection*{Question 3 (5 marks):}
What is the \textbf{AndroidManifest.xml}?

\subsection*{Question 4 (5 marks):}
Carefully describe the \textbf{Android Framework}.

\subsection*{Question 5 (3 marks):}
Describe the process in creating a new \textbf{locale} in an application.

\subsection*{Question 6 (5 marks):}
What is the importance of setting up permissions, i.e., location \& write storage when developing applications?

\subsection*{Question 7 (5 marks):}
Describe the process of populating a \textbf{Spinner} widget with data from \textbf{strings.xml}.

\subsection*{Question 8 (3 marks):}
Describe the purpose of \textbf{searchable.xml}.

\subsection*{Question 9 (6 marks):}
What is a \textbf{Parcelable} class \& what is it used for? Carefully describe the difference between a \textbf{Parcelable} \& \textbf{Parcelize}.

\subsection*{Question 10 (5 marks):}
Describe the process in creating a custom \textbf{Toast}.

\subsection*{Question 11 (6 marks)}
\textbf{Android} architecture is made up of four components. Carefully describe what they are \& provide two examples for each component.

\subsection*{Question 12 (5 marks)}
Fill in the blank. \_\_\_\_\_\_\_\_\_\_\ is an interface that provides random access (read/write) to the result set returned by a query.

\subsection*{Question 13 (3 marks)}
Describe the difference between the \textbf{ACCESS\_COARSE\_LOCATION} permission \& \textbf{ACCESS\_FINE\_LOCATION} permission.

\subsection*{Question 14 (6 marks):}
When an \textbf{asynchronous task} is executed, the task goes through four steps. Carefully describe what each step is.

\subsection*{Question 15 (5 marks):}
What is the purpose of each of the four default \textbf{Android} resource directories?

\subsection*{Question 16 (5 marks):}
What is a \textbf{RecyclerView}? Explain the process of setting up a \textbf{RecyclerView} \& instantiating it in an application.

\subsection*{Question 17 (5 marks):}
Describe what the \textbf{RecyclerView.Adapter} public method \textbf{notifyDataSetChanged} does.

\subsection*{Question 18 (5 marks):}
Specific to \textbf{Kotlin}, what is a \textbf{companion object}?

\subsection*{Question 19 (5 marks):}
What are the different storage methods in \textbf{Android}? Carefully describe what they are \& provide one example for each method.

\subsection*{Question 20 (10 marks):}
You are required to write an application to allow people to track a fleet of fishing boats. The boats periodically report their status (type(s) \& total weight(s) of fish caught, \& geolocation of the boat) to a central server. Your application will make a request to this server every 30 minutes \& receive a response in \textbf{JSON} format. The user of the application will be able to look at (a) a list of boats \& their accompanying data, \& (b) a map that shows the position of each boat. \\

Describe the class \& methods you will need to make this application work, including the essential methods \& callbacks. Explain the purpose of each.

\end{document}