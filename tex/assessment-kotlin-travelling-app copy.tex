% Author: Grayson Orr

\documentclass{article}
\author{}

\usepackage{graphicx}
\usepackage{wrapfig}
\usepackage{enumerate}
\usepackage{hyperref}
\usepackage{color, soul}
\usepackage{alltt}
\usepackage[margin = 2.25cm]{geometry}
\usepackage[table]{xcolor}
\usepackage{fancyhdr}
\hypersetup{
  colorlinks = true,
  urlcolor = blue
}
\setlength\parindent{0pt}
\pagestyle{fancy}
\fancyhf{}
\rhead{College of Engineering, Construction and Living Sciences\\Bachelor of Information Technology}
\lfoot{Kotlin Travelling App\\Version 1, Semester Two, 2020}
\rfoot{\thepage}

\begin{document}

\begin{figure}
    \centering
    \includegraphics[width=50mm]{./img/logo.png}
\end{figure}

\title{College of Engineering, Construction and Living Sciences\\Bachelor of Information Technology\\IN721: Design and Development of Applications for Mobile Devices\\Level 7, Credits 15\\\textbf{Kotlin Travelling App}}
\date{}
\maketitle

\section*{Assessment Overview}
For this individual or group assessment, you will develop \& publish a travelling application using Kotlin in Android Studio \& Google Play Store. 
Intermediate Android features such as asynchronous tasks, recycler view widget, card view widget, shared preferences, SQLite \& maps were covered formally in class. The main purpose of this assessment is not just to build a mobile application, rather to demonstrate your ability to effectively learn more intermediate \& advanced Android features independently. In addition, marks will be allocated for application robustness, code elegance, documentation \& git usage. \\

The travelling application will help you sound like a local abroad \& help you adapt to a new culture. You begin by selecting a continent \& country tool. For example, if you are travelling to Japan, you will have text translation support, text-to-speech support, a selection of key Japanese phrases \& an interactive quiz to test your knowledge of Japanese culture. A user of your travelling application should be able to select from at least two country tools per continent \textbf{excluding} Antarctica.

\section*{Assessment Table}
\renewcommand{\arraystretch}{1.5}
\begin{tabular}{|l|l|l|l|l|}
    \hline      
    \vtop{\hbox{\strut \textbf{Assessment}}\hbox{\strut \textbf{Activity}}} & \textbf{Weighting} & \vtop{\hbox{\strut \textbf{Learning}}\hbox{\strut \textbf{Outcomes}}} & \vtop{\hbox{\strut \textbf{Assessment}}\hbox{\strut \textbf{Grading Scheme}}} & \vtop{\hbox{\strut \textbf{Completion}}\hbox{\strut \textbf{Requirements}}} \\
                            
    \hline
                                
    \small Practicals                                                       & \small 10\%        & \small 1, 3, 4                                                        & \small CRA                                                                    & \small Cumulative                                                           \\ \hline
    \small Kotlin Travelling App                                            & \small 35\%        & \small 1, 3, 4                                                        & \small CRA                                                                    & \small Cumulative                                                           \\ \hline
    \small React Native Hacker News App                                     & \small 25\%        & \small 1, 3, 4                                                        & \small CRA                                                                    & \small Cumulative                                                           \\ \hline   
    \small Kotlin Exam                                    & \small 15\%        & \small 2, 3, 4                                                        & \small CRA                                                                    & \small Cumulative                                                           \\ \hline   
    \small React Native Exam                                     & \small 15\%        & \small 2, 3, 4                                                        & \small CRA                                                                    & \small Cumulative                                                           \\ \hline   
\end{tabular}

\section*{Conditions of Assessment}
You will complete this assessment outside timetabled class time, however, there will be availability during the teaching sessions to discuss the requirements and progress of this assessment. This assessment will need to be completed by Wednesday, 14 October 2020 at 5pm. 

\section*{Pass Criteria}
This assessment is criterion-referenced with a cumulative pass mark of 50\%.

\section*{Submission Details}
You must submit your program files via \textbf{GitHub Classroom}. Here is the link to the repository you will use for your submission – \href{https://classroom.github.com/a/Z5yfBBUS}{https://classroom.github.com/a/Z5yfBBUS}. \hl{The program files in the \textbf{master} branch will be used to run your application.}

\section*{Group Contribution}
\hl{All git commit messages must identify which member(s) participated in the associated work session. Proportional contribution will be determined by inspection of the commit logs. If the commit logs show evidence of significantly uneven contribution proportion, the lecturer may choose to adjust the mark of the lesser contributor downward by an amount derived from the individual contributions.}

\section*{Authenticity}
All parts of your submitted assessment must be completely your work and any references must be cited appropriately including, externally-sourced graphic elements. All media must be royalty free (or legally purchased) for educational use. Failure to do this will result in a mark of zero.

\section*{Policy on Submissions, Extensions, Resubmissions \& Resits}
The school's process concerning \textbf{Submissions, Extensions, Resubmissions and Resits} complies with Otago Polytechnic policies. Students can view policies on the Otago Polytechnic website located at \href{https://www.op.ac.nz/about-us/governance-and-management/policies}{https://www.op.ac.nz/about-us/governance-and-management/policies}.

\section*{Extensions}
Please familiarise yourself with the assessment due date. If you need an extension, please contact your lecturer before the due date. If you require more than a week's extension, a medical certificate or support letter from your manager may be needed.

\section*{Resubmissions}
Students may be requested to resubmit an assessment following a rework of part/s of the original assessment. Resubmissions are to be completed within a negotiable short time frame and usually must be completed within the timing of the course to which the assessment relates. Resubmissions will be available to students who have made a genuine attempt at the first assessment opportunity \hl{\& achieved a D grade (40-49\%)}. The maximum grade awarded for resubmission will be C-.

\section*{Learning Outcomes}
At the successful completion of this course, students will be able to:
\begin{enumerate}
    \item Implement complete, non-trivial, industry-standard mobile applications following sound architectural and code-quality standards.
    \item Explain relevant principles of human perception and cognition and their importance to software design.
    \item Identify relevant use cases for a mobile computing scenario and incorporate them into an effective user experience design.
    \item Follow industry standard software engineering practice in the design of mobile applications.
\end{enumerate}

\newpage

\section*{Instructions} 

\subsection*{Functionality \& Robustness - Learning Outcomes 1, 3, 4}
\begin{itemize}
    \item Application must open without file structure modification in Android Studio.
    \item Application must run without code modification on multiple mobile devices.
    \item Text translation support for at least two countries per continent. If a country is multilingual (use of more than one language), choose only one language. 
    \begin{itemize}
        \item Use an asynchronous task \& Yandex Translate API to translate text from one language to another.
        \item Display a progress dialog while text is being translated. 
        \item Handle incorrectly formatted input fields, for example, a blank or empty edit text widget.
        \item \textbf{Resource:} \footnotesize\href{https://tech.yandex.com/translate}{https://tech.yandex.com/translate}
    \end{itemize}
    \item Text-to-speech support for at least two countries per continent. 
    \begin{itemize}
        \item If a country is not supported, please handle correctly by either using a snackbar widget or a toast message.
        \item \textbf{Resource:} \footnotesize\href{https://developer.android.com/reference/kotlin/android/speech/tts/TextToSpeech}{https://developer.android.com/reference/kotlin/android/speech/tts/TextToSpeech} 
    \end{itemize}
    \item Selection of at least four key phrases per country per continent.
    \item An interactive quiz for at least one country per continent.
    \begin{itemize}
        \item Quiz topics may include animals, culture, food \& drink, geography \& sport.
        \item Each quiz must have at least five questions.
        \item Questions can be multi-choice \&/or true or false.
        \item Each question must have an image. Multi-choice questions must have four answers.
        \item Display appropriate feedback for correct \& incorrect answers by either using a text view widget or a toast message. If a question is answered incorrectly, display the correct answer.
        \item At the end of each quiz, store the user's score in a SQLite database table. 
        \item For each quiz, display the user's top three scores in either a recycler view widget or a text view widget.
    \end{itemize}
    \item Localization support for at least two countries per continent.  
    \begin{itemize}
        \item \textbf{Resource:} \footnotesize\href{https://developer.android.com/guide/topics/resources/localization/}{https://developer.android.com/guide/topics/resources/localization}
    \end{itemize}
    \item Application can be exited via a dialog fragment. The dialog fragment should prompt the user when the user double taps the mobile device's back button.
    \item For each country, display a Google marker on a Google map. The marker's information window should display the name of the country \& capital city. The marker's coordinates will be the latitude \& longitude of the capital city.
    \begin{itemize}
        \item \textbf{Resource:} \footnotesize\href{https://developers.google.com/maps/documentation/android-sdk/start}{https://developers.google.com/maps/documentation/android-sdk/start}
    \end{itemize}
    \item Switch widget which toggles between light \& dark mode. 
    \begin{itemize}
        \item The state (true or false) of the switch widget must be stored in shared preferences.
        \item The mode will be based off the state of the switch widget, for example, true equals dark mode \& false equals light mode.
    \end{itemize}
    \begin{itemize}
        \item \textbf{Resource:} \footnotesize\href{https://developer.android.com/guide/topics/ui/look-and-feel/darktheme}{https://developer.android.com/guide/topics/ui/look-and-feel/darktheme}
    \end{itemize}
    \item Splash screen with an image view widget \& transition animation.
    \begin{itemize}
        \item The transition animation must be a custom animation XML file.
        \item \textbf{Resource:} \footnotesize\href{https://developer.android.com/guide/topics/resources/animation-resource}{https://developer.android.com/guide/topics/resources/animation-resource}
    \end{itemize}
    \item Adaptive launcher icon which displays a variety of shapes across different mobile devices.
    \begin{itemize}
        \item \textbf{Resources:}
        \begin{itemize}
            \item \footnotesize\href{https://developer.android.com/guide/practices/ui\_guidelines/icon\_design\_adaptive}{https://developer.android.com/guide/practices/ui\_guidelines/icon\_design\_adaptive}
            \item \footnotesize\href{https://romannurik.github.io/AndroidAssetStudio/icons-launcher.html}{https://romannurik.github.io/AndroidAssetStudio/icons-launcher.html} 
        \end{itemize}
    \end{itemize}
    \item Bottom navigation view widget which navigates the user to the appropriate activities.
    \begin{itemize}
        \item \textbf{Resource:} \footnotesize\href{https://developer.android.com/reference/com/google/android/material/bottomnavigation/BottomNavigationView}{https://developer.android.com/reference/com/google/android/material/bottomnavigation/BottomNavigationView}
    \end{itemize}
    \item Display the application's privacy policy in a web view widget.
    \begin{itemize}
        \item \textbf{Resource:} \footnotesize\href{https://play.google.com/about/developer-content-policy}{https://play.google.com/about/developer-content-policy}
    \end{itemize}
    \item Visually attractive user-interface with a coherent graphical theme \& style using Material Design.
    \item Application is published to Google Play Store.
    \begin{itemize}
        \item When you create the application, please name the package appropriately, for example, \\ \textbf{op.johndoe.travelling}. \textbf{Note:} replace \textbf{johndoe} with your Otago Polytechnic Ltd username.
        \item \textbf{Resources:}
        \begin{itemize}
            \item \footnotesize\href{https://support.google.com/googleplay/android-developer/answer/113469?hl=en}{https://support.google.com/googleplay/android-developer/answer/113469?hl=en}
            \item \footnotesize\href{https://developer.android.com/studio/publish/app-signing}{https://developer.android.com/studio/publish/app-signing}
        \end{itemize}
    \end{itemize}
    \item Ability to download the application from Google Play Store on to multiple mobile devices.
\end{itemize}

\subsection*{Documentation \& Git Usage - Learning Outcomes 3, 4}
\begin{itemize}
    \item Provide the following in the repository README file:
    \begin{itemize}
        \item Privacy policy which discloses user information collected by the application.
        \item Sketched wireframes of the application. This can be hand-written or digitalised.
        \item Step-by-step user guide detailing each activity. The user guide must contain a screenshot of each activity in the application.
        \item Commented code is documented using KDoc \& generated to \textbf{Markdown} using Dokka.
        \begin{itemize}
            \item \textbf{Resources:}
            \begin{itemize}
                \item \footnotesize\href{https://kotlinlang.org/docs/reference/kotlin-doc.html}{https://kotlinlang.org/docs/reference/kotlin-doc.html}
                \item \footnotesize\href{https://github.com/Kotlin/dokka}{https://github.com/Kotlin/dokka}
            \end{itemize}
        \end{itemize} 
    \end{itemize}
    \item At least 10 feature branches excluding the \textbf{master} branch.
    \begin{itemize}
        \item Your branches must be prefix with \textbf{feature}, for example, \textbf{feature-$<$name of functional requirement$>$}.
        \item For each branch, merge your own pull request to the \textbf{master} branch.
    \end{itemize}
    \item Commit messages must reflect the context of each functional requirement change. \textbf{Note:} 
\end{itemize}




































\end{document}