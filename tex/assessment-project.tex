% Author: Grayson Orr
% Course: IN721: Mobile Application Development

\documentclass{article}
\author{}

\usepackage{graphicx}
\usepackage{wrapfig}
\usepackage{enumerate}
\usepackage{hyperref}
\usepackage[margin = 2.25cm]{geometry}
\usepackage[table]{xcolor}
\usepackage{fancyhdr}
\hypersetup{
  colorlinks = true,
  urlcolor = blue
}
\setlength\parindent{0pt}
\pagestyle{fancy}
\fancyhf{}
\rhead{College of Engineering, Construction and Living Sciences\\Bachelor of Information Technology}
\lfoot{Project\\Version 2, Semester One, 2021}
\rfoot{\thepage}

\begin{document}

\begin{figure}
	\centering
	\includegraphics[width=50mm]{./img/logo.png}
\end{figure}

\title{College of Engineering, Construction and Living Sciences\\Bachelor of Information Technology\\IN721: Mobile Application Development\\Level 7, Credits 15\\\textbf{Project}}
\date{}
\maketitle

\section*{Assessment Overview}
In this assessment, you will develop \& publish a travelling application using \textbf{Kotlin} in \textbf{Android Studio} \& \textbf{Google Play Store}. \textbf{Android} features such as \textbf{ViewModel}, \textbf{LiveData}, \textbf{Room Database} \& \textbf{Google Map} were formally covered in the teaching sessions. The main purpose of this assessment is not just to build a mobile application, rather to demonstrate your ability to effectively learn intermediate/advanced \textbf{Android} \& application development features independently. In addition, marks will be allocated for code elegance, documentation \& \textbf{Git/GitHub} usage. \\

The travelling application will help you sound like a local abroad \& help you adapt to a new culture. You will begin by selecting a \href{https://www.worldometers.info/geography/7-continents/}{continent} \& country tool. For example, if you were to travel to Spain, you would be provided with text translation \& text to speech support, a selection of key Spanish phrases, an interactive quiz to test your knowledge of Spanish culture \& a map containing locations of top-rated tourist Spanish attractions. A user of your travelling application must be able to select from at least two country tools per continent \textbf{excluding} Antarctica.

\section*{Learning Outcomes}
At the successful completion of this course, learners will be able to:
\begin{enumerate}
	\item Implement \& publish complete, non-trivial, industry-standard mobile applications following sound architectural \& code-quality standards.
	\item Identify relevant use cases for a mobile computing scenario \& incorporate them into an effective user experience design.
	\item Follow industry standard software engineering practice in the design of mobile applications.
\end{enumerate}

\section*{Assessment Table}
\renewcommand{\arraystretch}{1.5}
\begin{tabular}{|l|l|l|l|l|}
	\hline
	\vtop{\hbox{\strut \textbf{Assessment}}\hbox{\strut \textbf{Activity}}} & \textbf{Weighting} & \vtop{\hbox{\strut \textbf{Learning}}\hbox{\strut \textbf{Outcomes}}} & \vtop{\hbox{\strut \textbf{Assessment}}\hbox{\strut \textbf{Grading Scheme}}} & \vtop{\hbox{\strut \textbf{Completion}}\hbox{\strut \textbf{Requirements}}} \\

	\hline

	\small Practical                                                        & \small 20\%        & \small 2, 3                                                           & \small CRA                                                                    & \small Cumulative                                                           \\ \hline
	\small Project                                                          & \small 80\%        & \small 1, 2, 3                                                        & \small CRA                                                                    & \small Cumulative                                                           \\ \hline
\end{tabular}

\section*{Conditions of Assessment}
You will complete this assessment during your learner managed time, however, there will be availability during the teaching sessions to discuss the requirements \& your progress of this assessment. This assessment will need to be completed by \textbf{Tuesday, 22 June 2021 at 5:00 PM}.

\section*{Pass Criteria}
This assessment is criterion-referenced (CRA) with a cumulative pass mark of \textbf{50\%} over all assessments in \textbf{IN721: Mobile Application Development}.

\section*{Authenticity}
All parts of your submitted assessment must be completely your work \& any references must be cited appropriately including, externally-sourced graphic elements. Provide your references in a \textbf{README.md} file. All media must be royalty free (or legally purchased) for educational use. Failure to do this will result in a mark of \textbf{zero} for this assessment.

\section*{Policy on Submissions, Extensions, Resubmissions \& Resits}
The school's process concerning submissions, extensions, resubmissions \& resits complies with \textbf{Otago Polytechnic} policies. Learners can view policies on the \textbf{Otago Polytechnic} website located at \href{https://www.op.ac.nz/about-us/governance-and-management/policies}{https://www.op.ac.nz/about-us/governance-and-management/policies}.

\section*{Submissions}
You must submit all program files via \textbf{GitHub Classroom}. Here is the link to the repository you will use for your submission – \href{https://classroom.github.com/a/FWk\_XkTA}{https://classroom.github.com/a/FWk\_XkTA}. The latest program files in the \textbf{main} branch will be used to run your application. Late submissions will incur a \textbf{10\% penalty per day}, rolling over at \textbf{5:00 PM}.

\section*{Extensions}
Familiarise yourself with the assessment due date. If you need an extension, contact the course lecturer before the due date. If you require more than a week's extension, a medical certificate or support letter from your manager may be needed.

\section*{Resubmissions}
Learners may be requested to resubmit an assessment following a rework of part/s of the original assessment. Resubmissions are to be completed within a negotiable short time frame \& usually must be completed within the timing of the course to which the assessment relates. Resubmissions will be available to learners who have made a genuine attempt at the first assessment opportunity \& achieved a \textbf{D grade (40-49\%)}. The maximum grade awarded for resubmission will be \textbf{C-}.

\section*{Resits}
Resits \& reassessments are not applicable in \textbf{IN721: Mobile Application Development}. 

\section*{Post-Assessment Evaluation}
As part of this assessment, you will be required to organise a meeting with the course lecturer to discuss your feedback. The meeting will be conducted after you receive your assessment result. This post-assessment evaluation will \textbf{not} be graded.

\newpage

\section*{Instructions}
You will need to submit an application \& documentation that meet the following requirements:

\subsection*{Functionality - Learning Outcomes 1, 2, 3 (40\%)}
\begin{itemize}
	\item Application must open without file structure modification in \textbf{Android Studio}.
	\item Application must run without code modification on multiple mobile devices. The mobile devices used to test your application will be:
	      \begin{itemize}
		      \item Pixel 2 - 5.0"
		      \item Pixel XL - 5.5"
		      \item Pixel 3a XL - 6.0"
	      \end{itemize}
	\item Application must run on \textbf{API 28: Android 9.0 (Pie)}.
	\item Text translation support. If a country is multilingual (use of more than one language), choose only one language. For example, Canada's main languages are English \& French.
	      \begin{itemize}
		      \item Use \textbf{Retrofit} \& the \textbf{Yandex Translate API} to translate text from one language to another. To use the \textbf{Yandex Translate API}, you will need an API key. A key is available in the \textbf{Microsoft Teams} course channel, under the \textbf{Files} tab.
		      \item Display a custom \textbf{ProgressDialog} while the text is being translated.
		      \item Handle incorrectly formatted input fields. For example, an \textbf{EditText} is blank or empty.
		      \item \textbf{Resource:} \footnotesize\href{https://tech.yandex.com/translate}{https://tech.yandex.com/translate}
	      \end{itemize}
	\item Text to speech support.
	      \begin{itemize}
		      \item If a country is not supported, handle gracefully with a custom \textbf{Toast} message.
		      \item \textbf{Resource:} \footnotesize\href{https://developer.android.com/reference/kotlin/android/speech/tts/TextToSpeech}{https://developer.android.com/reference/kotlin/android/speech/tts/TextToSpeech}
	      \end{itemize}
	\item Selection of at least five key phrases. For example, "No worries, mate, she'll be right" is key phrase in Australia.
	\item An interactive quiz for each country.
	      \begin{itemize}
		      \item Quiz data must be fetched from a \textbf{Spring Boot REST API} using \textbf{Retrofit}.
		      \item Quiz topics may include animals, culture, food, drink, geography \& sport.
		      \item Each quiz must have at least five questions.
		      \item Questions are multi-choice \& true or false.
		      \item Multi-choice questions must have four answers.
		      \item Each question must have an image.
		      \item Each question must be answered within a \textbf{30 second} time limit.
		      \item Display appropriate feedback for correct \& incorrect answers with a custom \textbf{Toast} message. If a question is answered incorrectly, display the correct answer.
		      \item At the end of each quiz, store the highest score in a \textbf{Room Database} table.
		      \item For each quiz, display the highest score in a \textbf{TextView}.
		      \item \textbf{Resource:} \footnotesize\href{https://kotlinlang.org/docs/jvm-spring-boot-restful.html}{https://kotlinlang.org/docs/jvm-spring-boot-restful.html}
	      \end{itemize}
	\item Localization support for each country.
	      \begin{itemize}
		      \item \textbf{Resource:} \footnotesize\href{https://developer.android.com/guide/topics/resources/localization/}{https://developer.android.com/guide/topics/resources/localization}
	      \end{itemize}
	\item Application can be exited via a \textbf{DialogFragment}. The \textbf{DialogFragment} must prompt the user when the user double taps the mobile device's back button.
	\item Google Map displaying top-rated tourist attractions.
	      \begin{itemize}
		      \item Top-rated tourist attraction data must be fetched from a \textbf{Spring Boot REST API} using \textbf{Retrofit}.
		      \item Each data object will be represented by a marker on a \textbf{Google Map}.
		      \item The marker's information window must display the attraction's name \& coordinates (latitude \& longitude).
		      \item \textbf{Resource:} \footnotesize\href{https://developers.google.com/maps/documentation/android-sdk/start}{https://developers.google.com/maps/documentation/android-sdk/start}
	      \end{itemize}
	\item \textbf{Switch} which toggles between light \& dark mode.
	      \begin{itemize}
		      \item The state (true or false) value of the \textbf{Switch} must be stored in \textbf{SharedPreferences}.
		      \item The mode will be based off the state value of the \textbf{Switch}. For example, true equals dark mode \& false equals light mode.
		      \item \textbf{Resource:} \footnotesize\href{https://developer.android.com/guide/topics/ui/look-and-feel/darktheme}{https://developer.android.com/guide/topics/ui/look-and-feel/darktheme}
	      \end{itemize}
	\item Splash screen with an \textbf{ImageView} \& transition animation.
	      \begin{itemize}
		      \item The transition animation must be a custom animation \textbf{XML} file.
		      \item \textbf{Resource:} \footnotesize\href{https://developer.android.com/guide/topics/resources/animation-resource}{https://developer.android.com/guide/topics/resources/animation-resource}
	      \end{itemize}
	\item Adaptive launcher icon which displays a variety of shapes across different mobile devices.
	      \begin{itemize}
		      \item \textbf{Resources:}
		            \begin{itemize}
			            \item \footnotesize\href{https://developer.android.com/guide/practices/ui\_guidelines/icon\_design\_adaptive}{https://developer.android.com/guide/practices/ui\_guidelines/icon\_design\_adaptive}
			            \item \footnotesize\href{https://romannurik.github.io/AndroidAssetStudio/icons-launcher.html}{https://romannurik.github.io/AndroidAssetStudio/icons-launcher.html}
		            \end{itemize}
	      \end{itemize}
	\item \textbf{BottomNavigationView} which navigates the user to the appropriate activities. For example, a menu icon for translation support, text to speech support, etc.
	      \begin{itemize}
		      \item \textbf{Resource:} \footnotesize\href{https://developer.android.com/reference/com/google/android/material/bottomnavigation/BottomNavigationView}{https://developer.android.com/reference/com/google/android/material/bottomnavigation/BottomNavigationView}
	      \end{itemize}
	\item A clickable link to your application's privacy policy.
	      \begin{itemize}
		      \item \textbf{Resource:} \footnotesize\href{https://play.google.com/about/developer-content-policy}{https://play.google.com/about/developer-content-policy}
	      \end{itemize}
	\item Visually attractive UI with a coherent graphical theme \& style using \textbf{Material Design}.
	\item Application is published to \textbf{Google Play Store}.
	      \begin{itemize}
		      \item To published to \textbf{Google Play Store}, you will need a \textbf{Google Play Console} account. The account's credentials are available in the \textbf{Microsoft Teams} course channel, under the \textbf{Files} tab. The account will be available to all learners in the course. \textbf{Do not} disable any applications published on this account.
		      \item When you create your application, name the package appropriately. For example, \\ \textbf{op.mobile.app.dev.$<$username$>$.travelling}. \textbf{Note:} replace \textbf{username} with your \textbf{Otago Polytechnic} username.
		      \item \textbf{Resources:}
		            \begin{itemize}
			            \item \footnotesize\href{https://support.google.com/googleplay/android-developer/answer/113469?hl=en}{https://support.google.com/googleplay/android-developer/answer/113469?hl=en}
			            \item \footnotesize\href{https://developer.android.com/studio/publish/app-signing}{https://developer.android.com/studio/publish/app-signing}
		            \end{itemize}
	      \end{itemize}
	\item Ability to download your application from \textbf{Google Play Store} on to multiple mobile devices. The mobile devices used to download your application will be:
	      \begin{itemize}
		      \item Pixel 2 - 5.0"
		      \item Pixel XL - 5.5"
		      \item Pixel 3a XL - 6.0"
	      \end{itemize}
	\item At least \textbf{15} UI tests which verify that your application is functioning correctly.
\end{itemize}

\subsection*{Code Elegance - Learning Outcomes 1, 3 (40\%)}
\begin{itemize}
	\item \textbf{Kotlin} \& \textbf{XML} files contain no magic numbers/strings. Store the values in the appropriate \textbf{XML} files. For example, numbers must be stored in an integer or dimension file \& strings must be stored in a string file.
	\item Use of intermediate variables. No method calls as arguments.
	\item Idiomatic use of control flow, data structures \& other in-built functions.
	\item Code adheres to various object-oriented design principles. For example, \textbf{DRY} \& \textbf{SOLID}.
	\item Efficient algorithmic approach.
\end{itemize}

\subsection*{Documentation \& Git/GitHub Usage - Learning Outcomes 2, 3 (20\%)}
\begin{itemize}
	\item Provide the following in your repository \textbf{README Markdown} file:
	      \begin{itemize}
		      \item Privacy policy which discloses user information collected by your application. For example, does this application require storage access or internet?
		      \item Sketched wireframes of your application. The wireframes must be sketched/designed using online software. This must be completed before you start coding.
		            \begin{itemize}
			            \item \textbf{Resource}: \footnotesize\href{https://moqups.com}{https://moqups.com}
		            \end{itemize}
		      \item Step-by-step user guide detailing each screen. The user guide must contain a screenshot of each screen in your application.
		      \item Commented code is documented using \textbf{KDoc} \& generated to \textbf{Markdown} using \textbf{Dokka}.
		      \item \textbf{Resources:}
		            \begin{itemize}
			            \item \footnotesize\href{https://kotlinlang.org/docs/reference/kotlin-doc.html}{https://kotlinlang.org/docs/reference/kotlin-doc.html}
			            \item \footnotesize\href{https://github.com/Kotlin/dokka}{https://github.com/Kotlin/dokka}
		            \end{itemize}
		      \item \textbf{Spring Boot REST API} endpoints for quiz \& tourist-attraction features.
		      \item URL to your application on \textbf{Google Play Store}.
	      \end{itemize}
				\item Continuous integration using \textbf{GitHub Actions}.
	      \begin{itemize}
		      \item \textbf{YAML} file must be configured for UI tests and generating an APK.
		      \item \textbf{Resource:} \footnotesize\href{https://wkrzywiec.medium.com/github-actions-for-android-first-approach-f616c24aa0f9}{https://wkrzywiec.medium.com/github-actions-for-android-first-approach-f616c24aa0f9}
	      \end{itemize}
	\item At least \textbf{10} feature branches excluding the \textbf{main} branch.
	      \begin{itemize}
		      \item Your branches must be prefix with \textbf{feature}, for example, \textbf{feature-$<$name of functional requirement$>$}.
		      \item Code in the branch must relate to the \textbf{feature}.
		      \item Once you have completed a \textbf{feature}, create a pull request \& assign the \textbf{GitHub} user \textbf{grayson-orr} to a reviewer. \textbf{Do not} merge your own pull request.
		      \item 
	      \end{itemize}

	\item Commit messages must reflect the context of each functional requirement change. \textbf{Do not} rewrite your \textbf{Git} history. It is important that the course lecturer can see how you worked on your assessment over time.
	      \begin{itemize}
		      \item \textbf{Resource:} \footnotesize\href{https://www.freecodecamp.org/news/writing-good-commit-messages-a-practical-guide}{https://www.freecodecamp.org/news/writing-good-commit-messages-a-practical-guide}
	      \end{itemize}
\end{itemize}
\end{document}
